\section{Conclusion \& future work}\label{sec:conclusion}

In this paper we study a language for higher-order recursion \pcfc{} in the setting of synthetic domain theory. Our main contribution is an internal, cost-sensitive version of Plotkin's computational adequacy theorem for \pcfc{}. In particular, we define and relate a denotational model of \pcfc{} to a new dynamic semantics for \pcfc{} defined directly in terms of computation that is both natural and mathematically appealing. Here we suggest some ideas for future investigations. 

\emph{Internal \vs{} external adequacy.} 
In the same vein as the work of Simpson~\cite{simpson:1999,simpson:2004}, we are also interested in giving a logical characterization of when internal computational adequacy (with respect to the computational semantics of \pcfc{}) implies external adequacy. However it is not clear to us what would be an analogous condition to 1-consistency: internal notions such as the initial lift algebra $\omega$ and synthetic $\omega$-chains do not have natural external counterparts. As mentioned in \cref{sec:discussion}, a first step would be to develop a systematic understanding of the logical aspects of the initial lift algebra from an external point of view.

As mentioned in \cref{sec:discussion}, one way to obtain external adequacy would be to follow the approach of~\cite{simpson:2004} and define \pcfc{} and its computational semantics purely in terms of computational natural numbers. Alternatively, we may decide to assume Axiom N (see \cref{sec:discussion}), which would imply that the computational semantics coincides with ordinary operational semantics in the internal logic of the SDT topos. We do not expect Axiom N to hold in the model we construct in this paper (\cf{} van Oosten and Simpson~\cite{van-oosten-simpson:2000}), but it does not appear to be a limitation of the general approach to the model construction; indeed we believe it should be possible to start with a different domain-theoretic site such that the embedding into the resulting sheaf topos preserves countable coproducts, which would be enough to validate Axiom N. 

\emph{Cost and information order.}
As discussed in \cref{subsubsec:intrinsic-preorder}, we would like to combine and develop a practical theory for the interaction of the domain-theoretic information order with a cost preorder in the sense of~Grodin \etal{}, who developed a ``preorder'' version of SDT in which ``predomains'' are types equipped with a preorder. Following the approach of relative sheaf models of SDT, we conjecture that one may build a model of SDT that further incorporates an intrinsic preorder structure by starting with a domain-theoretic site internal to the category of \emph{simplicial sets}.%

\emph{Recursive types.} 
We have emphasized recursion at the term level, but synthetic domain theory is also compatible with having recursive types. Simpson~\cite{simpson:2004} has developed in a very general setting the theory and existence of algebraically compact categories of predomains in SDT, and we hope to instantiate the ideas of \opcit{} at a \emph {relative} sheaf model of SDT similar to the one presented in this paper.
