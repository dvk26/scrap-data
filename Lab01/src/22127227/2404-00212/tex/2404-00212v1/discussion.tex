\section{Discussion of related work}\label{sec:discussion}


\emph{Cost analysis in type theory.}
The original motivation for proving internal, cost-sensitive computational adequacy results grew out of the work of Niu \etal{}~\cite{niu-sterling-grodin-harper:2022} on formalizing cost analysis of functional programs in dependent type theory. Niu and Harper~\cite{niu-harper:2023} prove such an adequacy theorem for a variant of the Algol language featuring a notion of first-order recursion in the form of while loops. The purpose of the present paper is to generalize that result to account for higher-order recursion. In contrast to both prior works, we have de-emphasized the role of the call-by-push-value language and instead work directly with the internal language of the SDT topos. Because the model we construct in \cref{sec:model} can also be seen as a model for the type theories in both prior works, we expect that it would be routine to formalize our results in a call-by-push-value version of type theory as well. 

\emph{Cost-sensitive computational adequacy.}
The kind of cost-sensitive adequacy theorem we prove in this paper has been proved in a classic domain-theoretic setting by Kavvos \etal{}~\cite{kavvos-morehouse-licata-danner:2019}, and the general theory of computational adequacy for languages with algebraic effects has been developed by Plotkin and Power~\cite{plotkin-power:2002}. The main difference between our work and those mentioned is that we aim to prove adequacy results internally to a type theory equipped with a phase distinction (such as the one in Niu \etal{}~\cite{niu-sterling-grodin-harper:2022}). As argued in Niu and Harper~\cite{niu-harper:2023}, such internal adequacy theorems can be used as a basis for the validity of axiomatic cost analysis in these type theories. 

\emph{Relative sheaf models of SDT.}
As we have explained in \cref{subsubsec:SDT}, the main sheaf models of synthetic domain theory take the form of Grothendieck topoi over the category of sets. Meanwhile, the logic of phase distinctions finds its home in presheaf topoi in which one finds many distinct subterminal objects that are neither globally true nor globally false; because the category of sets is boolean and two-valued, it can have no non-trivial phase distinctions.
For this reason, Sterling and Harper~\cite{sterling-harper:2022} have proposed to combine synthetic domain theory with phase distinctions by developing models in \emph{relative} Grothendieck topoi~\cite{johnstone:2002} over a presheaf topos that exhibits a phase distinction. In other words, rather than building a site out of predomains in the category of sets, \opcit built an \emph{internal} site based on \emph{internal} predomains in a category of presheaves.
Our model of cost-sensitive synthetic domain theory is similar to that of Sterling and Harper~\cite{sterling-harper:2022}. On the other hand, our proof of computational adequacy is different from that of \opcit, as the latter contains a subtle error~\cite{sterling:2023:noninterference-erratum} involving a mismatch between the existential quantifier and the join of a family of $\Sigma$-propositions in the lifting of free algebras to formal approximation relations.


\emph{Computational adequacy in SDT.} 
Our approach to internal denotational semantics and computational adequacy of \pcfc{} builds on the pioneering work of Simpson~\cite{simpson:1999} on proving the computational adequacy property of \pcf{} in elementary topoi models of SDT. One difference between our work and that of \opcit{} is the addition of the phase distinction, which we use to give an intrinsic denotational account of the interaction between cost and behavior in \pcfc{}. Another difference is in the SDT axioms used and the ensuing definition of the dynamic semantics of the object programming language. Simpson~\cite{simpson:1999} assumes a property called \emph{Axiom N} that closes the dominance $\Sigma$ under countable joins of decidable families in the ambient logic, a fact that we do not rely on in our constructions. The benefit of this axiom is that it enables \opcit{} to give an internal definition of \pcf{} whose dynamic semantics can be characterized by means of existentially quantified statements of the form $\some{n : \Nat} \phi(n)$ where $\phi$ is a primitive recursive predicate. This is used to show that a general property of the internal logic of topoi called \emph{1-consistency}\footnote{A topos $\mathcal{E}$ is \emph{1-consistent} when a closed formula $\some{n : \Nat} \phi(n)$ of the form described above holding in the internal logic of $\CatIdent{E}$ implies that it holds externally.} is both necessary and sufficient to \emph{externalize} the internal adequacy proof into a corresponding proof in ordinary mathematics. In a follow-up paper, Simpson~\cite{simpson:2004} gave a different logical criterion for the equivalence of internal and external adequacy called \emph{computational 1-consistency} that does not rely on Axiom N. Roughly the idea is to define the programming language and its operational semantics in terms of the \emph{computational natural numbers} (analogous to the predomain $\NatP$ in this paper); computational 1-consistency is just the property needed to ensure that internal computational observations hold externally as well.

By contrast, the dynamic semantics of \pcfc{} in this paper is defined \emph{computationally} and is not known to be equivalent to the operational semantics of Simpson~\cite{simpson:1999} in the absence of Axiom N. However, we expect that a version of our computational semantics using the computational natural numbers will be equivalent to the semantics given in Simpson~\cite{simpson:2004}. On the other hand, we find the computational semantics developed in \cref{sec:comp-sem} both philosophically and mathematically compelling and deserving of further investigation in its own right. Moreover, although the computational semantics of \pcfc{} does not appear to be definable in terms of countable joins, it can be defined using \emph{synthetic} $\omega$-joins of decidable families. Therefore, we conjecture that one may externalize the internal adequacy proof of \pcfc{} in the manner of Simpson~\cite{simpson:1999} by developing a Kripke-Joyal semantics for the sheaf model defined in \cref{sec:model} that unfolds an internal statement involving synthetic $\omega$-joins to an external statement in the metatheory. 
