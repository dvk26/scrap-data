\section{Cost-aware denotational semantics of \texorpdfstring{\pcfc{}}{pcf/cost}}\label{sec:den-sem}

Both the syntax of \pcfc{} and our model construction is parameterized in a monoid object \(( \mathbb{C} , +, 0)\) representing the cost structure. We require the following conditions to hold: 
\begin{enumerate}
  \item{\emph{Computational}: \(\mathbb{C}\) is a predomain with \(\Sigma\)-equality. }  
  \item{\emph{Phase separation}: \(\mathbb{C}\) is purely intensional, \ie $(\P \to \mathbb{C}) \cong 1$. }  
\end{enumerate}

That $\mathbb{C}$ is computational are used in two places in the computational adequacy proof: when reasoning about the \emph{computational semantics} of \pcfc{} (see \cref{sec:comp-sem}), we need $\mathbb{C}$ to be discrete (which follows from \cref{prop:sigma-dec-disc}) in order to prove the property that sequential composition of computations may be decomposed (\cref{prop:comp-ind-seq}), and we need $\mathbb{C}$ to have $\Sigma$-equality when showing that the \emph{formal approximation predicates} (see \cref{fig:formal-approx-rel}) associated to semantic domains are admissible. We require $\mathbb{C}$ to be purely intensional to ensure that the denotational semantics of \pcfc{} exhibits an \emph{intrinsic} noninterference property of cost and behavior as sketched in \cref{subsec:den-sem}. As discussed in \cref{subsec:predomains}, there is a canonical way of turning any monoid $\mathcal{M}$ into a purely intensional type by means of the sealing monad $\P \vee -$. 

\subsection{The partial cost monad}

To model partiality and cost as a single effect, the computation types of \pcfc{} are interpreted as algebras for the monad $(\mathbb{T}, \eta_\mathbb{T}, \mu_\mathbb{T})$ whose action on points is defined by composing the lift and the writer monad $\kw{T} A =  \mathsf{L} ( \mathbb{C}   \times  A)$. The distributive law for $\kw{T}$ and the resulting monad structure is displayed in \cref{fig:partial-cost-monad}, where we write $x \leftarrow_\mathbb{M} e; f(x)$ for the induced bind operation of a monad $\mathbb{M}$ where $f : A \to \kw{M}(B)$ is a map into a free $\mathbb{M}$-algebra. We will also write $f^\sharp(e)$ for sequencing $e : \kw{M}(A)$ and $f : A \to X$ for an $\mathbb{M}$-algebra $X$. 
\begin{figure}
  \begin{multicols}{2}
    \iblock{
    \mrow{\tau : \mathbb{C} \times \lift{A} \to \lift{(\mathbb{C} \times A)}}
    \mrow{\tau(c, ( \phi , f))  =  ( \phi ,  \lambda  u: \phi .~(c, f u))}
    }
  \columnbreak
  \iblock{
    \mrow{\eta_\mathbb{T}(a) = \eta_\kw{L}(0, a)}
    \mrow{\mu_\mathbb{T}(e) = (c, x) \leftarrow_\mathbb{L} e; (c', a) \leftarrow_\mathbb{L} x; (c + c', a)}
  }
  \end{multicols}
  \caption{Left: distributive law; right: monad structure of $\mathbb{T}$.}
  \label{fig:partial-cost-monad}
\end{figure}

\subsection{The derived cost algebra}

To model the cost effect $\kw{step} : \mathbb{C} \times X \to X$, we use the fact that every algebra for $\mathbb{T}$ is canonically an algebra for the writer monad $\mathbb{C} \times -$, which is a general property of composite monads defined from a distributive law~\cite[Section 2]{beck_distributive_1969}. In our case this means that every $\mathbb{T}$-algebra has an underlying cost algebra as well; we write $\costmap : \mathbb{C} \times X \to X$ for the cost algebra map. 

\begin{restatable}{proposition}{PropCostAlgLaws}\label[prop]{prop:cost-alg-laws}
  The action of the derived cost algebra satisfies the following equations for $e : \kw{T}(A)$ and $f : A \to X$ for some $\mathbb{T}$-algebra $X$: 
  \[\begin{aligned}   
    &c  \boxplus_{\kw{T}(A)} e = ( e { \downarrow } ,  \lambda  u.~ \mu _{ \mathbb{C}   \times  -}(c, e)) \\   
    &f^\sharp(c  \boxplus_{\kw{T}(A)}  e) = c  \boxplus_X  (f^\sharp~e) \\   
    &(c  \boxplus_{A \to X}  f)~a = c  \boxplus_{X}  (f~a) 
  \end{aligned}\]
\end{restatable}

\subsection{Denotational semantics of \texorpdfstring{\pcfc{}}{pcf/cost}}

The semantics of \pcfc{} is based around the free-forgetful adjunction associated with the partial cost monad in which value types are predomains and computation types are $\mathbb{T}$-algebra valued in predomains.
The essential parts of the model is displayed in \cref{fig:model}. Most of the type structure of \pcfc{} is interpreted using the cartesian closed structure of predomains; note that the numerals type is interpreted as the natural numbers type of predomains $\NatP$, which is not the same as the ambient natural numbers $\Nat$.  

\begin{figure}
\begingroup
\setlength\columnsep{-6.5cm}
\begin{multicols}{2}
  \iblock{
    \mrow{\sem{-} : \tpv \to \mathcal{U}_\kw{predom}}
    \mrow{\sem{-} : \tpc \to \kw{Alg}_\kw{\mathbb{T}}(\mathcal{U}_\kw{predom})}
    \row
    \mrow{\sem{\F{A}} = \kw{T}(\sem{A})}
    \mrow{\sem{\U{X}} = U(\sem{X})}
    \mrow{\sem{1} = 1}
    \mrow{\sem{\kw{ans}} = 2}
    \mrow{\sem{\kw{nat}} = \NatP}
    \mrow{\sem{A \rightharpoonup X} = \sem{A} \to \sem{X}}
  }
  \columnbreak
  \iblock{
    \mrow{\sem{-} : \impl{\Gamma, A} (\Gamma \vdash A) \to \sem{\Gamma} \to \sem{A}}
    \mrow{\sem{-} : \impl{\Gamma, A} (\Gamma \vdash X) \to \sem{\Gamma} \to U(\sem{X})}
    \row
    \mrow{\sem{\kw{ret}(a)}(\gamma) = \eta_\mathbb{T}(\sem{a}(\gamma))}
    \mrow{\sem{\kw{step}(c, e)}(\gamma) = c \costmap \sem{e}(\gamma)}
    \mrow{\sem{\kw{bind}(e, f)}(\gamma) = \kw{bind}(\sem{e}(\gamma), \lambda a.~\sem{f}(a, \gamma))}
    \mrow{\sem{\kw{fix}(f)}(\gamma) = \kw{fix}(\lambda x.~\sem{f}(x, \gamma))}
  }
\end{multicols}
\endgroup
\caption{Selected clauses of the model; in the above we write $U(X)$ for the carrier of a $\mathbb{T}$-algebra $X$.}
\label{fig:model}
\end{figure}
