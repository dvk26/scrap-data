\section{\texorpdfstring{\pcfc{}}{pcf/cost}: a language for cost-aware higher-order recursion}\label{sec:pcfc}

Our main technical result is the computational adequacy property for \pcfc{}, a version of Plotkin's \pcf{}~\cite{plotkin:1977} equipped with an abstract cost effect. The treatment of both cost structure and recursion as a call-by-push-value effect in \pcfc{} is inspired by~\cite{kavvos-morehouse-licata-danner:2019}. The syntax of \pcfc{} is parameterized in a monoid $\mathbb{C}$ representing the cost structure; in \cref{sec:den-sem} we will impose further properties on $\mathbb{C}$ when we define the denotational semantics of \pcfc{}. As mentioned in \cref{subsec:den-sem}, the type structure of \pcfc{} is generated by a pair of operators $\kw{F}, \kw{U}$ that corresponds semantically to the free-forgetful adjunction between plain sets and the category of \emph{partial cost algebras}, which are sets equipped with an action for the partial cost monad $\lift{(\mathbb{C} \times -)}$. The types and terms of \pcfc{} are summarized in \cref{fig:pcfc}, defined as inductive definitions in the SDT topos \CatIdent{E}. 

\begin{figure}
\begingroup
\setlength\columnsep{-6.5cm}
\begin{multicols}{2}
  \iblock{
    \mhang{
      \textbf{inductive}~\kw{Ty}^+ : \kw{Set}~\textbf{where}
    }{
      \mrow{\kw{ans} : \kw{Ty}^+}
      \mrow{\kw{nat} : \kw{Ty}^+}
      \mrow{\kw{U} : \kw{Ty}^\ominus \to \kw{Ty}^+}
    }
    \row
    \mhang{
      \textbf{inductive}~\kw{Ty}^\ominus : \tpv~\textbf{where}
    }{
      \mrow{\kw{F} : \kw{Ty}^+ \to \kw{Ty}^\ominus}
      \mrow{{\rightharpoonup} :\kw{Ty}^+ \to \kw{Ty}^\ominus \to \kw{Ty}^\ominus}
    }
    \row 
    \mrow{
      \kw{Tm}^\ominus : \kw{Con} \to \kw{Ty}^\ominus \to \tpv
    }
    \mrow{
      \kw{Tm}^\ominus(\Gamma, X) = \kw{Tm}^+(\Gamma, \kw{U}(X))
    }
  }
  \columnbreak 
  \iblock{
    \mhang{
      \textbf{inductive}~\kw{Tm}^+ : \kw{Con} \to \kw{Ty}^+ \to \kw{Set}~\textbf{where}
    }{
      \mrow{\kw{var} : \kw{Var}(\Gamma, A) \to \kw{Tm}(\Gamma, A)}
      \mrow{\kw{yes} : \kw{Tm}^+(\Gamma, \kw{ans})}
      \mrow{\kw{no} : \kw{Tm}^+(\Gamma, \kw{ans})}
      \mrow{\kw{zero} : \kw{Tm}^+(\Gamma, \kw{nat})}
      \mrow{\kw{succ} : \kw{Tm}^+(\Gamma, \kw{nat}) \to \kw{Tm}^+(\Gamma, \kw{nat})}
      \mrow{\kw{ap} : \kw{Tm}^\ominus(\Gamma, A \rightharpoonup X) \to \kw{Tm}^+(\Gamma, A) \to \kw{Tm}^\ominus(X)}
      \mrow{\kw{ret} : \kw{Tm}^+(\Gamma, A) \to \kw{Tm}^\ominus(\Gamma, \kw{F}(A))}
      \mrow{\kw{step} : \mathbb{C} \to \kw{Tm}^\ominus(\Gamma, X) \to \kw{Tm}^\ominus(\Gamma, X)}
      \mrow{\kw{bind} : \kw{Tm}^\ominus(\Gamma, \kw{F}(A)) \to \kw{Tm}^\ominus(A::\Gamma, X) \to \kw{Tm}^\ominus(\Gamma, X)}
      \mrow{\kw{ifz} : \kw{Tm}^+(\Gamma, \kw{nat}) \to \kw{Tm}^\ominus(\Gamma, X) \to \kw{Tm}^\ominus(\kw{nat}::\Gamma, X) \to \kw{Tm}^\ominus(\Gamma, X)}
      \mrow{\kw{fix} : \kw{Tm}^\ominus(\kw{U}(X)::\Gamma, X) \to \kw{Tm}^\ominus(\Gamma, X)}
      \mrow{\kw{lam} : \kw{Tm}^\ominus(A::\Gamma, X) \to \kw{Tm}^\ominus(\Gamma, A \rightharpoonup X)}
    }
  }
\end{multicols}
\endgroup
\caption{The grammar of types and terms in \pcfc. We will often omit $\kw{Tm}^{\{+,\ominus\}}$ in the case of closed terms and write $\Gamma \vdash \{A, X\}$ for the type $\kw{Tm}^{\{+, \ominus\}}(\Gamma, \{A, X\})$. Given $e : \U{\F{A}}$ and $f : \U{(A \to X)}$, we also write $e; f$ for $\kw{bind}(e, f) : \U{X}$. }
\label{fig:pcfc}
\end{figure}


