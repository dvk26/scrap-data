
\section{An SDT model of the intension-extension phase distinction}\label{sec:model}

To obtain a model for the constructions of the preceding sections, we instantiate the sheaf model of Sterling and Harper~\cite{sterling-harper:2022} at the poset \(\I  =  { \left \{ \kw{ext}  \le  \kw{int} \right \} }\) representing the intension-extension security order. The basic idea is to first develop domain theory \emph{internal} to the presheaf topos $\PSH{\I}$, from which we may obtain an appropriate \emph{internal} domain-theoretic site that embeds into a sheaf topos model of SDT in the sense of \cref{def:axioms}. The reason to consider internal sites is that we may build into the base category the intension-extension phase distinction that is preserved through the embedding. 

In \opcit{} the internal domain theory of $\PSH{\I}$ is developed in terms of constructive dcpos following the work de Jong~\cite{dejong:2023:thesis}. These internal dcpos are similar to ordinary dcpos; for example, one may use them to give the denotational semantics of \textbf{PCF}~\cite{dejong:2023:thesis}. The difference lies in the dominance $\Sigma$ of the category of internal dcpos given by the subobject classifier $\Omega_{\PSH{\I}}$: while a partial element of ordinary domains in $\textbf{Set}$ is either defined or not, a partial element of a domain internal to $\PSH{\I}$ may have the phase proposition $\P$ as its support, where $\P$ is the intermediate proposition in $\Omega_{\PSH{\I}}$. Recalling the interpretation of a map $A \to \Sigma$ as a computational predicate, this also means that predicates can be phase-dependent in the sense of holding only at the extensional phase. 

We recall some definitions from Sterling and Harper~\cite{sterling-harper:2022} (SH22). 

\begin{definition}
  A \emph{Scott-open immersion} of a dcpo is any mono $U \mono A$ arising from a predicate $A \to \Sigma$. 
\end{definition}

\begin{definition}
  In a category, a \emph{sink} on an object $A$ is a set of morphisms into $A$.  
\end{definition}

\begin{definition}
  In a category with pullbacks, a \emph{Cartesian coverage} is an assignment of objects $A$ to set of sinks on $A$ that is stable under pullback. 
\end{definition}

\begin{definition}
  The \emph{finite open cover topology} is generated by the Cartesian coverage assigning to each dcpo $A$ the set of sinks $\{U_i \mono A\}_i$ on $A$ with every $U_i \mono A$ a Scott-open immersion and $\sup_i U_i \cong A$. 
\end{definition}

Our domain-theoretic site is given by an internal category \(\CatIdent{C}\) of small dcpos in $\PSH{\I}$. We embed \(\CatIdent{C}\) into a Grothendieck topos \Sh{\CatIdent{C}}, obtained by localizing $\PSH{\CatIdent{C}}$ at the finite open cover topology. The purpose of this localization is to ensure that the finite joins of the dominance in \(\CatIdent{C}\) are preserved by the embedding into \Sh{\CatIdent{C}}. This property was notably used by \opcit to implement the semantics of termination declassification; here we use the finite join structure of \(\Sigma\) to show that \(2\) is an extensional predomain. The phase distinction in \Sh{\CatIdent{C}} is inherited from the ambient presheaf topos, where it is represented by $\kw{ext}$. 

\begin{theorem}[SH22, Corollary 90]\label{thm:representable-predomains}
  Every representable presheaf is well-complete.  
\end{theorem}

Thus we obtain a functor $y :  \CatIdent{C} \hookrightarrow \Sh{\CatIdent{C}}$ restricting the Yoneda embedding $\y{\CatIdent{C}} : \CatIdent{C} \hookrightarrow \PSH{\CatIdent{C}}$ onto sheaves. 

\begin{theorem}[SH22, B.1.4.3]
  The representable $\Sigma = y(\Sigma)$ is a dominance in \Sh{\CatIdent{C}}.  
\end{theorem}

\begin{theorem}[SH22, Corollary 79]
  Coproducts in \CatIdent{C} are given by unions of families of Scott-opens.  
\end{theorem}

\begin{corollary}\label{cor:coproducts-sheaves}
  Finite coproducts of \CatIdent{C} are preserved by the embedding into sheaves. 
\end{corollary}

\begin{theorem}[SH22, Axiom-SDT-1]\label{thm:dominance-finite-join}
  The dominance $\Sigma$ has finite joins that are preserved by the inclusion $\Sigma \hookrightarrow \Omega$.
\end{theorem}

\begin{restatable}{theorem}{ThmModel}\label[thm]{thm:model}
  Setting $\Sigma = y(\Omega_{\PSH{\I}})$ and $\P = y(\y{\I}(\kw{ext}))$, we have that $(\Sh{\CatIdent{C}}, \Sigma, \P)$ is an SDT model of the intension-extension phase distinction in the sense of \cref{def:axioms}. 
\end{restatable}

\begin{proof}
  By \cref{thm:representable-predomains} we know that $\Sigma$ is a well-complete dominance in $\Sh{\CatIdent{C}}$. To show that \Sh{\CatIdent{C}} also models the intension-extension phase distinction, we observe that the presheaf model of the intension-extension phase distinction of Niu~\etal~\cite{niu-sterling-grodin-harper:2022} restricts to a smaller model in \(\CatIdent{C}\): every subterminal object in \PSH{\I} is an internal dcpo, and so we may take the subterminal \(\P  =  \y{\I}(\kw{ext})\) to be the phase proposition in \(\CatIdent{C}\). The phase proposition \(\P =  y(\P) \) in \(\Sh{\CatIdent{C}}\) is classified by $\Sigma$: since $y$ is fully faithful, every $\phi : \Sigma$ arise from a unique map $1_{\PSH{\I}} \to \Omega_{\PSH{\I}}$. 
  
  Moreover, we can directly verify that \(2_{\CatIdent{C}} = 2_{\PSH{\I}} = 1_{\PSH{\I}} + 1_{\PSH{1}}\) is internally orthogonal to \(\P\) in $\CatIdent{C}$. By \cref{cor:coproducts-sheaves}, $y$ preserves finite coproducts, so we have that \(2_{ \Sh{\CatIdent{C}} } =  y(2_\CatIdent{C})\). We observe that 2 is extensional because the restricted embedding \(y :  \CatIdent{C}   \to   \Sh{\CatIdent{C}}\) is both full and faithful and preserves products.
  To see that \(2\) is replete, we observe that 2 is isomorphic to its type of singletons:
  \[\begin{aligned}     
    2  \cong   { \left \{ \phi  :  \Sigma ^2  \mid   { \left ( \all{a, b : 2}   \phi ~a  \land   \phi ~b  \to  a = b \right )}   \land   { \left ( \phi ( \kw{inl} \cdot   \star )  \lor   \phi ( \kw{inr}   \cdot   \star ) \right )} \right \} }    
  \end{aligned}\]
  Because \(\Sigma\) is closed under finite joins (by \cref{thm:dominance-finite-join}), the type of singletons of 2 can be defined as the limit of a diagram of replete types, and so it is replete as well. 
  
  Lastly, to see that Phoa's principle is satisfied, we note that it holds in \(\CatIdent{C}\). Since the subobject \({ \left \{ ( \phi ,  \psi ) :  \Sigma   \times   \Sigma   \mid   \phi   \to   \psi \right \} }\) can be defined as the equalizer of \(\pi _1 :  \Sigma   \times   \Sigma   \to   \Sigma\) and \(\land  :  \Sigma   \times   \Sigma   \to   \Sigma\), every object in the diagram is defined using the cartesian closed structure of \(\CatIdent{C}\), it is preserved by any cartesian closed embedding, and so Phoa's principle also holds in \Sh{\CatIdent{C}}.
\end{proof}
