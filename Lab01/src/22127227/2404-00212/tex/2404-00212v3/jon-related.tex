\emph{Relative sheaf models of SDT.}
As we have explained in \cref{subsubsec:SDT}, the main sheaf models of synthetic domain theory take the form of Grothendieck topoi over the category of sets. Meanwhile, the logic of phase distinctions finds its home in presheaf topoi in which one finds many distinct subterminal objects that are neither globally true nor globally false; because the category of sets is boolean and two-valued, it can have no non-trivial phase distinctions.
For this reason, Sterling and Harper~\cite{sterling-harper:2022} have proposed to combine synthetic domain theory with phase distinctions by developing models in \emph{relative} Grothendieck topoi~\cite{johnstone:2002} over a presheaf topos that exhibits a phase distinction. In other words, rather than building a site out of predomains in the category of sets, \opcit built an \emph{internal} site based on \emph{internal} predomains in a category of presheaves.
Our model of cost-sensitive synthetic domain theory is similar to that of Sterling and Harper~\cite{sterling-harper:2022}. On the other hand, our proof of computational adequacy is different from that of \opcit, as the latter contains a subtle error~\cite{sterling:2023:noninterference-erratum} involving a mismatch between the existential quantifier and the join of a family of $\Sigma$-propositions in the lifting of free algebras to formal approximation relations.
