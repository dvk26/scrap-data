\begin{filecontents}{main.aux}
\relax
\providecommand\hyper@newdestlabel[2]{}
\providecommand\HyperFirstAtBeginDocument{\AtBeginDocument}
\HyperFirstAtBeginDocument{\ifx\hyper@anchor\@undefined
\global\let\oldnewlabel\newlabel
\gdef\newlabel#1#2{\newlabelxx{#1}#2}
\gdef\newlabelxx#1#2#3#4#5#6{\oldnewlabel{#1}{{#2}{#3}}}
\AtEndDocument{\ifx\hyper@anchor\@undefined
\let\newlabel\oldnewlabel
\fi}
\fi}
\global\let\hyper@last\relax
\gdef\HyperFirstAtBeginDocument#1{#1}
\providecommand\HyField@AuxAddToFields[1]{}
\providecommand\HyField@AuxAddToCoFields[2]{}
\citation{BN13}
\citation{Stanley,Bell}
\citation{Knuth}
\citation{Henzinger07}
\citation{Darwish18}
\citation{VBHLRS99,BCMSV05,BC09,BLMV11,fronhofer2013random}
\citation{EM11a,EM11b,EMS20}
\citation{EM11a}
\citation{EM11a,EM11b,EMM13,EMS20}
\citation{EM11a}
\citation{KMSS14,CM15,CS15,Reuveni16,BBR16,PR17,CS18,RMR19,PP19b,MBMS20,SB21,MBM22,BMS23,BMMS23}
\citation{PKR20,BS20a,DBM23,SBEM23,MOK23}
\citation{SBEM23}
\citation{SM}
\newlabel{FirstPage}{{}{1}{}{section*.1}{}}
\@writefile{toc}{\contentsline {title}{Minimizing the Profligacy of Searches with Reset}{1}{section*.2}\protected@file@percent }
\@writefile{toc}{\contentsline {abstract}{Abstract}{1}{section*.1}\protected@file@percent }
\newlabel{profligacy}{{1}{1}{}{equation.0.1}{}}
\citation{EM11a}
\citation{BMS22}
\citation{BMS22}
\citation{PKR20,BS20a,BS20b}
\citation{FPSRR20,BBPMC20,MOK23}
\citation{SBEM23}
\citation{EM11a}
\citation{BMS22}
\citation{SM}
\citation{AW06}
\newlabel{C}{{2}{2}{}{equation.0.2}{}}
\@writefile{lof}{\contentsline {figure}{\numberline {1}{\ignorespaces  Schematic trajectory for RBM (red) and RBB (blue). For RBM, the particle diffuses without constraint while for RBB the particle is constrained to return to the starting position at the completion time $t_f$. The vertical arrows represent resetting events. The searcher is considered to have found the target if it crosses $x=m$ in any of its excursions before $t_f$. \relax }}{2}{figure.caption.3}\protected@file@percent }
\providecommand*\caption@xref[2]{\@setref\relax\@undefined{#1}}
\newlabel{fig:schem}{{1}{2}{Schematic trajectory for RBM (red) and RBB (blue). For RBM, the particle diffuses without constraint while for RBB the particle is constrained to return to the starting position at the completion time $t_f$. The vertical arrows represent resetting events. The searcher is considered to have found the target if it crosses $x=m$ in any of its excursions before $t_f$. \relax }{figure.caption.3}{}}
\newlabel{lin_RBM}{{3}{2}{}{equation.0.3}{}}
\newlabel{quad_RBM}{{4}{2}{}{equation.0.4}{}}
\newlabel{phit_RBM}{{5}{2}{}{equation.0.5}{}}
\newlabel{fig:RBM_linear}{{2a}{3}{RBM Linear Cost\relax }{figure.caption.4}{}}
\newlabel{sub@fig:RBM_linear}{{a}{3}{RBM Linear Cost\relax }{figure.caption.4}{}}
\newlabel{fig:RBM_quad}{{2b}{3}{RBM Quadratic Cost\relax }{figure.caption.4}{}}
\newlabel{sub@fig:RBM_quad}{{b}{3}{RBM Quadratic Cost\relax }{figure.caption.4}{}}
\newlabel{fig:RBB_linear}{{2c}{3}{RBB Linear Cost\relax }{figure.caption.4}{}}
\newlabel{sub@fig:RBB_linear}{{c}{3}{RBB Linear Cost\relax }{figure.caption.4}{}}
\newlabel{fig:RBB_quad}{{2d}{3}{RBB Quadratic Cost\relax }{figure.caption.4}{}}
\newlabel{sub@fig:RBB_quad}{{d}{3}{RBB Quadratic Cost\relax }{figure.caption.4}{}}
\@writefile{lof}{\contentsline {figure}{\numberline {2}{\ignorespaces Phase diagrams: heat map of $R^*$ which minimizes profligacy $(\xi )$ for 4 different cases: (a) RBM with linear cost (b) RBM with quadratic cost (c) RBB with linear cost and (d) RBB with quadratic cost. The phase boundaries delineate the boundary between regions of zero and non-zero $R^*$: a full line indicates a continuous transition and broken line a discontinuous transition. In (a) the horizontal red line indicates the threshold, $M_{\rm  T}$, above which there is no transition.\relax }}{3}{figure.caption.4}\protected@file@percent }
\newlabel{fig:transitions}{{2}{3}{Phase diagrams: heat map of $R^*$ which minimizes profligacy $(\xi )$ for 4 different cases: (a) RBM with linear cost (b) RBM with quadratic cost (c) RBB with linear cost and (d) RBB with quadratic cost. The phase boundaries delineate the boundary between regions of zero and non-zero $R^*$: a full line indicates a continuous transition and broken line a discontinuous transition. In (a) the horizontal red line indicates the threshold, $M_{\rm T}$, above which there is no transition.\relax }{figure.caption.4}{}}
\newlabel{lin_RBB}{{6}{3}{}{equation.0.6}{}}
\newlabel{quad_RBB}{{7}{3}{}{equation.0.7}{}}
\newlabel{phit_RBB}{{8}{3}{}{equation.0.8}{}}
\citation{BMMS23}
\citation{PP19b,BMMS23}
\citation{SM}
\citation{Goldenfeld}
\newlabel{fig:RBM_overhang_transition_1}{{3a}{4}{\relax }{figure.caption.5}{}}
\newlabel{sub@fig:RBM_overhang_transition_1}{{a}{4}{\relax }{figure.caption.5}{}}
\newlabel{fig:RBM_overhang_transition_2}{{3b}{4}{\relax }{figure.caption.5}{}}
\newlabel{sub@fig:RBM_overhang_transition_2}{{b}{4}{\relax }{figure.caption.5}{}}
\newlabel{fig:tricritical_transition}{{3c}{4}{\relax }{figure.caption.5}{}}
\newlabel{sub@fig:tricritical_transition}{{c}{4}{\relax }{figure.caption.5}{}}
\newlabel{fig:overhang_transition}{{3d}{4}{\relax }{figure.caption.5}{}}
\newlabel{sub@fig:overhang_transition}{{d}{4}{\relax }{figure.caption.5}{}}
\@writefile{lof}{\contentsline {figure}{\numberline {3}{\ignorespaces Optimal resetting rate $R^*$ versus $\lambda $ for values of $M$ in the regions where continuous and discontinuous transitions are in close proximity. Panels (a) and (b) RBM with quadratic cost: there are two different ranges of $M$ (see Fig.\nobreakspace  {}\ref {fig:transitions}) where on increasing $\lambda $, there is first a continuous transition followed by a discontinuous transition. The discontinuous jump in $R^*$ closes at a non-zero value of $R^*$ as $M$ is varied and thus is not a usual tricritical point. Panel (c) RBB with linear cost: we have a usual tricritical point where the jump in discontinuity closes at $R^* = 0$. Panel (d) RBB with quadratic cost: similar to RBM the jump closes at non-zero value of $R^*$.\relax }}{4}{figure.caption.5}\protected@file@percent }
\newlabel{fig:tricritical_overhang}{{3}{4}{Optimal resetting rate $R^*$ versus $\lambda $ for values of $M$ in the regions where continuous and discontinuous transitions are in close proximity. Panels (a) and (b) RBM with quadratic cost: there are two different ranges of $M$ (see Fig.~\ref {fig:transitions}) where on increasing $\lambda $, there is first a continuous transition followed by a discontinuous transition. The discontinuous jump in $R^*$ closes at a non-zero value of $R^*$ as $M$ is varied and thus is not a usual tricritical point. Panel (c) RBB with linear cost: we have a usual tricritical point where the jump in discontinuity closes at $R^* = 0$. Panel (d) RBB with quadratic cost: similar to RBM the jump closes at non-zero value of $R^*$.\relax }{figure.caption.5}{}}
\newlabel{landau_expansion}{{9}{4}{}{equation.0.9}{}}
\citation{Redner}
\bibdata{mainNotes,main}
\bibcite{BN13}{{1}{2013}{{Bressloff\ and\ Newby}}{{}}}
\bibcite{Stanley}{{2}{2011}{{Viswanathan\ \emph  {et~al.}}}{{Viswanathan, Da~Luz, Raposo,\ and\ Stanley}}}
\bibcite{Bell}{{3}{2012}{{Bell}}{{}}}
\bibcite{Knuth}{{4}{1998}{{Knuth}}{{}}}
\bibcite{Henzinger07}{{5}{2007}{{Henzinger}}{{}}}
\bibcite{Darwish18}{{6}{2018}{{Darwish}}{{}}}
\bibcite{VBHLRS99}{{7}{1999}{{Viswanathan\ \emph  {et~al.}}}{{Viswanathan, Buldyrev, Havlin, da~Luz, Raposo,\ and\ Stanley}}}
\bibcite{BCMSV05}{{8}{2005}{{B{\'e}nichou\ \emph  {et~al.}}}{{B{\'e}nichou, Coppey, Moreau, Suet,\ and\ Voituriez}}}
\bibcite{BC09}{{9}{2009}{{Bartumeus\ and\ Catalan}}{{}}}
\bibcite{BLMV11}{{10}{2011}{{B{\'e}nichou\ \emph  {et~al.}}}{{B{\'e}nichou, Loverdo, Moreau,\ and\ Voituriez}}}
\bibcite{fronhofer2013random}{{11}{2013}{{Fronhofer\ \emph  {et~al.}}}{{Fronhofer, Hovestadt,\ and\ Poethke}}}
\bibcite{EM11a}{{12}{2011{}}{{Evans\ and\ Majumdar}}{{}}}
\bibcite{EM11b}{{13}{2011{}}{{Evans\ and\ Majumdar}}{{}}}
\bibcite{EMS20}{{14}{2020}{{Evans\ \emph  {et~al.}}}{{Evans, Majumdar,\ and\ Schehr}}}
\bibcite{EMM13}{{15}{2013}{{Evans\ \emph  {et~al.}}}{{Evans, Majumdar,\ and\ Mallick}}}
\bibcite{KMSS14}{{16}{2014}{{Ku\'smierz\ \emph  {et~al.}}}{{Ku\'smierz, Majumdar, Sabhapandit,\ and\ Schehr}}}
\bibcite{CM15}{{17}{2015}{{Campos\ and\ M{\'e}ndez}}{{}}}
\bibcite{CS15}{{18}{2015}{{Christou\ and\ Schadschneider}}{{}}}
\bibcite{Reuveni16}{{19}{2016}{{Reuveni}}{{}}}
\bibcite{BBR16}{{20}{2016}{{Bhat\ \emph  {et~al.}}}{{Bhat, De~Bacco,\ and\ Redner}}}
\bibcite{PR17}{{21}{2017}{{Pal\ and\ Reuveni}}{{}}}
\bibcite{CS18}{{22}{2018}{{Chechkin\ and\ Sokolov}}{{}}}
\bibcite{RMR19}{{23}{2019}{{Ray\ \emph  {et~al.}}}{{Ray, Mondal,\ and\ Reuveni}}}
\bibcite{PP19b}{{24}{2019}{{Pal\ and\ Prasad}}{{}}}
\bibcite{MBMS20}{{25}{2020}{{Mercado-Vásquez\ \emph  {et~al.}}}{{Mercado-Vásquez, Boyer, Majumdar,\ and\ Schehr}}}
\bibcite{SB21}{{26}{2021}{{Schumm\ and\ Bressloff}}{{}}}
\bibcite{MBM22}{{27}{2022}{{Mercado-Vásquez\ \emph  {et~al.}}}{{Mercado-Vásquez, Boyer,\ and\ Majumdar}}}
\bibcite{BMS23}{{28}{2023}{{Biroli\ \emph  {et~al.}}}{{Biroli, Majumdar,\ and\ Schehr}}}
\bibcite{BMMS23}{{29}{2024}{{Boyer\ \emph  {et~al.}}}{{Boyer, Mercado-V{\'a}squez, Majumdar,\ and\ Schehr}}}
\bibcite{PKR20}{{30}{2020}{{Pal\ \emph  {et~al.}}}{{Pal, Ku\'smierz,\ and\ Reuveni}}}
\bibcite{BS20a}{{31}{2020{}}{{Bodrova\ and\ Sokolov}}{{}}}
\bibcite{DBM23}{{32}{2023}{{Bruyne\ and\ Mori}}{{}}}
\@writefile{toc}{\contentsline {section}{\numberline {}Acknowledgments}{5}{section*.6}\protected@file@percent }
\@writefile{toc}{\contentsline {section}{\numberline {}References}{5}{section*.7}\protected@file@percent }
\bibcite{SBEM23}{{33}{2023}{{Sunil\ \emph  {et~al.}}}{{Sunil, Blythe, Evans,\ and\ Majumdar}}}
\bibcite{MOK23}{{34}{2023}{{Mori\ \emph  {et~al.}}}{{Mori, Olsen,\ and\ Krishnamurthy}}}
\bibcite{SM}{{35}{}{{SM}}{{}}}
\bibcite{BMS22}{{36}{2022}{{De~Bruyne\ \emph  {et~al.}}}{{De~Bruyne, Majumdar,\ and\ Schehr}}}
\bibcite{BS20b}{{37}{2020{}}{{Bodrova\ and\ Sokolov}}{{}}}
\bibcite{FPSRR20}{{38}{2020}{{Tal-Friedman\ \emph  {et~al.}}}{{Tal-Friedman, Pal, Sekhon, Reuveni,\ and\ Roichman}}}
\bibcite{BBPMC20}{{39}{2020}{{Besga\ \emph  {et~al.}}}{{Besga, Bovon, Petrosyan, Majumdar,\ and\ Ciliberto}}}
\bibcite{AW06}{{40}{2006}{{Abate\ and\ Whitt}}{{}}}
\bibcite{Goldenfeld}{{41}{2018}{{Goldenfeld}}{{}}}
\bibcite{Redner}{{42}{2001}{{Redner}}{{}}}
\bibstyle{apsrev4-2}
\citation{REVTEX42Control}
\citation{apsrev42Control}
\newlabel{LastBibItem}{{42}{6}{}{section*.7}{}}
\newlabel{LastPage}{{}{6}{}{}{}}
\gdef \@abspage@last{6}

\relax
\providecommand\hyper@newdestlabel[2]{}
\providecommand\HyperFirstAtBeginDocument{\AtBeginDocument}
\HyperFirstAtBeginDocument{\ifx\hyper@anchor\@undefined
\global\let\oldnewlabel\newlabel
\gdef\newlabel#1#2{\newlabelxx{#1}#2}
\gdef\newlabelxx#1#2#3#4#5#6{\oldnewlabel{#1}{{#2}{#3}}}
\AtEndDocument{\ifx\hyper@anchor\@undefined
\let\newlabel\oldnewlabel
\fi}
\fi}
\global\let\hyper@last\relax
\gdef\HyperFirstAtBeginDocument#1{#1}
\providecommand\HyField@AuxAddToFields[1]{}
\providecommand\HyField@AuxAddToCoFields[2]{}
\newlabel{FirstPage}{{}{1}{}{Doc-Start}{}}
\@writefile{toc}{\contentsline {title}{Minimizing the Profligacy of Searches with Reset\\[1ex] Supplemental Material}{1}{section*.1}\protected@file@percent }
\@writefile{toc}{\contentsline {section}{\numberline {I}Derivation of the mean cost}{1}{section*.2}\protected@file@percent }
\newlabel{sec:cost}{{I}{1}{}{section*.2}{}}
\newlabel{expC}{{S1}{1}{}{equation.1.1}{}}
\newlabel{G0}{{S2}{1}{}{equation.1.2}{}}
\newlabel{renew1}{{S3}{1}{}{equation.1.3}{}}
\@writefile{toc}{\contentsline {paragraph}{\numberline {}Note on notation}{1}{section*.3}\protected@file@percent }
\newlabel{K}{{S6}{2}{}{equation.1.6}{}}
\newlabel{Cinv}{{S9}{2}{}{equation.1.9}{}}
\newlabel{numinv}{{S10}{2}{}{equation.1.10}{}}
\@writefile{toc}{\contentsline {subsection}{\numberline {A}Resetting Brownian Motion (RBM)}{2}{section*.4}\protected@file@percent }
\newlabel{costn}{{S12}{2}{}{equation.1.12}{}}
\newlabel{Crbm}{{S13}{2}{}{equation.1.13}{}}
\newlabel{Crbmlin}{{S15}{2}{}{equation.1.15}{}}
\newlabel{Crbmquad}{{S16}{2}{}{equation.1.16}{}}
\citation{EM11a}
\citation{BMS22}
\@writefile{toc}{\contentsline {subsection}{\numberline {B}Resetting Brownian Bridge (RBB)}{3}{section*.5}\protected@file@percent }
\newlabel{norm_time}{{S19}{3}{}{equation.1.19}{}}
\newlabel{Crbblin}{{S23}{3}{}{equation.1.23}{}}
\newlabel{Crbbquad}{{S24}{3}{}{equation.1.24}{}}
\@writefile{toc}{\contentsline {section}{\numberline {II}Derivation of Success Probabilities}{3}{section*.7}\protected@file@percent }
\newlabel{sec:sprob}{{II}{3}{}{section*.7}{}}
\@writefile{toc}{\contentsline {subsection}{\numberline {A}Resetting Brownian Motion (RBM)}{3}{section*.8}\protected@file@percent }
\newlabel{Prbm}{{S26}{3}{}{equation.2.26}{}}
\citation{AW06}
\providecommand*\caption@xref[2]{\@setref\relax\@undefined{#1}}
\newlabel{fig:linear_costs}{{S1a}{4}{\relax }{figure.caption.6}{}}
\newlabel{sub@fig:linear_costs}{{a}{4}{\relax }{figure.caption.6}{}}
\newlabel{fig:quad_costs}{{S1b}{4}{\relax }{figure.caption.6}{}}
\newlabel{sub@fig:quad_costs}{{b}{4}{\relax }{figure.caption.6}{}}
\@writefile{lof}{\contentsline {figure}{\numberline {S1}{\ignorespaces Comparison of mean total cost $\langle C \rangle $ for (a) linear cost and (b) quadratic cost per reset for RBM and RBB. It can be seen that the costs increase monotonically for both the cases. This is consistent with the intuition that more frequent resets will incur a higher cost. Although in the case of a linear cost the mean cost increases indefinitely with $R$ we find that in the case of a quadratic cost the mean cost saturates as $R\to \infty $.\relax }}{4}{figure.caption.6}\protected@file@percent }
\newlabel{fig:costs}{{S1}{4}{Comparison of mean total cost $\langle C \rangle $ for (a) linear cost and (b) quadratic cost per reset for RBM and RBB. It can be seen that the costs increase monotonically for both the cases. This is consistent with the intuition that more frequent resets will incur a higher cost. Although in the case of a linear cost the mean cost increases indefinitely with $R$ we find that in the case of a quadratic cost the mean cost saturates as $R\to \infty $.\relax }{figure.caption.6}{}}
\@writefile{toc}{\contentsline {subsection}{\numberline {B}Resetting Brownian Bridge (RBB)}{4}{section*.9}\protected@file@percent }
\newlabel{Prbb}{{S27}{4}{}{equation.2.27}{}}
\@writefile{toc}{\contentsline {section}{\numberline {III}Landau-like expansion for Profligacy ($\xi $)}{4}{section*.11}\protected@file@percent }
\newlabel{sec:landau}{{III}{4}{}{section*.11}{}}
\newlabel{landau_expansion_S}{{S29}{4}{}{equation.3.29}{}}
\newlabel{fig:ps_m_1}{{S2a}{5}{\relax }{figure.caption.10}{}}
\newlabel{sub@fig:ps_m_1}{{a}{5}{\relax }{figure.caption.10}{}}
\newlabel{fig:ps_m_0_5}{{S2b}{5}{\relax }{figure.caption.10}{}}
\newlabel{sub@fig:ps_m_0_5}{{b}{5}{\relax }{figure.caption.10}{}}
\@writefile{lof}{\contentsline {figure}{\numberline {S2}{\ignorespaces Comparison of probability of finding the target $(P_{\rm  s})$ as a function $R$ for (a) $M = 1.0$ and (b) $M = 0.5$. For RBM (purple), $P_{\rm  s}$ can initially be a decreasing of increasing function function depending on the value of $M$, whereas for RBB (green) $P_{\rm  s}$ always increases initially with resetting, reaches a maximum then decreases for large $R$.\relax }}{5}{figure.caption.10}\protected@file@percent }
\newlabel{fig:ps}{{S2}{5}{Comparison of probability of finding the target $(P_{\rm s})$ as a function $R$ for (a) $M = 1.0$ and (b) $M = 0.5$. For RBM (purple), $P_{\rm s}$ can initially be a decreasing of increasing function function depending on the value of $M$, whereas for RBB (green) $P_{\rm s}$ always increases initially with resetting, reaches a maximum then decreases for large $R$.\relax }{figure.caption.10}{}}
\newlabel{p_hit_RBM2}{{S30}{5}{}{equation.3.30}{}}
\newlabel{p_hit_RBB2}{{S31}{5}{}{equation.3.31}{}}
\newlabel{phit_expansion_RBM_S}{{S32}{5}{}{equation.3.32}{}}
\newlabel{phit_expansion_RBB_S}{{S36}{5}{}{equation.3.36}{}}
\citation{Redner}
\citation{BMS22}
\citation{Goldenfeld}
\@writefile{toc}{\contentsline {subsection}{\numberline {A}Classical continuous transition}{6}{section*.12}\protected@file@percent }
\newlabel{RBM_continious_transition}{{S45}{6}{}{equation.3.45}{}}
\newlabel{RBB_continuous_transition}{{S46}{6}{}{equation.3.46}{}}
\newlabel{fig:c_c_1}{{S3a}{7}{\relax }{figure.caption.13}{}}
\newlabel{sub@fig:c_c_1}{{a}{7}{\relax }{figure.caption.13}{}}
\newlabel{fig:c_c_2}{{S3b}{7}{\relax }{figure.caption.13}{}}
\newlabel{sub@fig:c_c_2}{{b}{7}{\relax }{figure.caption.13}{}}
\@writefile{lof}{\contentsline {figure}{\numberline {S3}{\ignorespaces The figure presents classical continuous transition observed in $\xi $ for the case of RBB with linear cost per reset when $\lambda $ is varied for $M=1.0$. (a) The global minimum which is at $R^* = 0$ initially continuously transitions to the new global minimum $R^*$ in (b) as $\lambda $ is varied.\relax }}{7}{figure.caption.13}\protected@file@percent }
\newlabel{fig:c_c}{{S3}{7}{The figure presents classical continuous transition observed in $\xi $ for the case of RBB with linear cost per reset when $\lambda $ is varied for $M=1.0$. (a) The global minimum which is at $R^* = 0$ initially continuously transitions to the new global minimum $R^*$ in (b) as $\lambda $ is varied.\relax }{figure.caption.13}{}}
\@writefile{toc}{\contentsline {subsection}{\numberline {B}Classical discontinuous transition}{7}{section*.14}\protected@file@percent }
\@writefile{toc}{\contentsline {subsection}{\numberline {C}Non-classical discontinuous transition}{7}{section*.16}\protected@file@percent }
\newlabel{fig:c_dc_1}{{S4a}{8}{\relax }{figure.caption.15}{}}
\newlabel{sub@fig:c_dc_1}{{a}{8}{\relax }{figure.caption.15}{}}
\newlabel{fig:c_dc_2}{{S4b}{8}{\relax }{figure.caption.15}{}}
\newlabel{sub@fig:c_dc_2}{{b}{8}{\relax }{figure.caption.15}{}}
\@writefile{lof}{\contentsline {figure}{\numberline {S4}{\ignorespaces The figure presents classical discontinuous transition observed in $\xi $ for the case of RBB with linear cost per reset when $\lambda $ is varied for $M=0.63$. (a) The global minimum which is at $R^* = 0$ initially discontinuously transitions to the new global minimum $R^*$ in (b) when the local minima becomes the new global minima as $\lambda $ is varied.\relax }}{8}{figure.caption.15}\protected@file@percent }
\newlabel{fig:c_dc}{{S4}{8}{The figure presents classical discontinuous transition observed in $\xi $ for the case of RBB with linear cost per reset when $\lambda $ is varied for $M=0.63$. (a) The global minimum which is at $R^* = 0$ initially discontinuously transitions to the new global minimum $R^*$ in (b) when the local minima becomes the new global minima as $\lambda $ is varied.\relax }{figure.caption.15}{}}
\newlabel{fig:nc_dc_1}{{S5a}{8}{\relax }{figure.caption.17}{}}
\newlabel{sub@fig:nc_dc_1}{{a}{8}{\relax }{figure.caption.17}{}}
\newlabel{fig:nc_dc_2}{{S5b}{8}{\relax }{figure.caption.17}{}}
\newlabel{sub@fig:nc_dc_2}{{b}{8}{\relax }{figure.caption.17}{}}
\@writefile{lof}{\contentsline {figure}{\numberline {S5}{\ignorespaces The figure presents the non-classical discontinuous transition occurs observed in RBB with quadratic cost per reset as $\lambda $ is varied for $M = 0.707$. (a) The global minimum initially emerges continuously from $R = 0$ as $\lambda $ is varied. (b) As lambda is further increased, the second minimum becomes the global minima and there is a discontinuous transition of the optimal value $R^*$ to the new global minimum.\relax }}{8}{figure.caption.17}\protected@file@percent }
\newlabel{fig:nc_dc}{{S5}{8}{The figure presents the non-classical discontinuous transition occurs observed in RBB with quadratic cost per reset as $\lambda $ is varied for $M = 0.707$. (a) The global minimum initially emerges continuously from $R = 0$ as $\lambda $ is varied. (b) As lambda is further increased, the second minimum becomes the global minima and there is a discontinuous transition of the optimal value $R^*$ to the new global minimum.\relax }{figure.caption.17}{}}
\bibdata{SupplementalNotes}
\bibstyle{apsrev4-2}
\citation{REVTEX42Control}
\citation{apsrev42Control}
\@writefile{toc}{\contentsline {section}{\numberline {IV}Further interpretation of the profligacy function}{9}{section*.18}\protected@file@percent }
\newlabel{sec:prof}{{IV}{9}{}{section*.18}{}}
\newlabel{weight.1}{{S48}{9}{}{equation.4.48}{}}
\newlabel{LastPage}{{}{9}{}{}{}}
\gdef \@abspage@last{9}

\end{filecontents}
