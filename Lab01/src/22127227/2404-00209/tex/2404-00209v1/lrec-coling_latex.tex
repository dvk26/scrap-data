% LREC-COLING 2024 Example; 
% LREC Is now using templates similar to the ACL ones. 
\documentclass[10pt, a4paper]{article}

% \usepackage[review]{lrec-coling2024} % this is the new style
\usepackage{lrec-coling2024} % this is the new style


% Standard package includes
\usepackage{times}
\usepackage{latexsym}
\usepackage[T1]{fontenc}

% This assumes your files are encoded as UTF8
\usepackage[utf8]{inputenc}

% This is not strictly necessary, and may be commented out,
% but it will improve the layout of the manuscript,
% and will typically save some space.
\usepackage{microtype}
\usepackage{inconsolata}
\usepackage{multirow}
\usepackage{amsmath}
% Include the graphicx
\usepackage{graphicx}
\usepackage{hyperref}

% If the title and author information does not fit in the area allocated, uncomment the following
%
% \setlength\titlebox{10cm}
%
% and set <dim> to something 5cm or larger.



\title{\methodname: Narrative Reasoning by Grounding to Eventuality-centric Knowledge Graphs}
% \title{Interpretable Narrative Reasoning by Grounding to Event-centric Knowledge Graphs}
% \title{Contextualized Narrative Reasoning by Grounding to Eventuality-centric Knowledge Graphs}

% \name{Author1, Author2, Author3} 

% \address{Affiliation1, Affiliation2, Affiliation3 \\
%          Address1, Address2, Address3 \\
%          author1@xxx.yy, author2@zzz.edu, author3@hhh.com\\
%          \{author1, author5, author9\}@abc.org\\}

\newcommand{\ust}{\ensuremath{^\spadesuit}}
\newcommand{\amazon}{\ensuremath{^\diamondsuit}}

\name{Cheng Jiayang\ust, Lin Qiu\amazon, Chunkit Chan\ust, Xin Liu\ust, Yangqiu Song\ust, Zheng Zhang\amazon}

\address{\ust The Hong Kong University of Science and Technology, 
\amazon Amazon AWS AI\\
         \{jchengaj, yqsong\}@cse.ust.hk, 
         zhaz@amazon.com}

% \ust The Hong Kong University of Science and Technology \\
% \westlake Westlake University \ \ \ \
% \amazon Amazon AWS AI\\
% \texttt{\{jchengaj, yqsong\}@cse.ust.hk} \ \ \ \ 
% \texttt{zhangyue@westlake.edu.cn} \ \ \ \ 
% \texttt{zhaz@amazon.com} \\ \\




\abstract{
Narrative reasoning relies on the understanding of eventualities in story contexts, which requires a wealth of background world knowledge.
To help machines leverage such knowledge, existing solutions can be categorized into two groups.
Some focus on implicitly modeling eventuality knowledge by pretraining language models (LMs) with eventuality-aware objectives.
However, this approach breaks down knowledge structures and lacks interpretability.
Others explicitly collect world knowledge of eventualities into structured eventuality-centric knowledge graphs (KGs).
However, existing research on leveraging these knowledge sources for free-texts is limited.
In this work, we propose an initial comprehensive framework called \methodname, which aims to tackle the problem of grounding free-texts to eventuality-centric KGs for contextualized narrative reasoning.
We identify two critical problems in this direction: the \textit{event representation} and \textit{sparsity} problems.
We provide simple yet effective parsing and partial information extraction methods to tackle these problems.
Experimental results demonstrate that our approach consistently outperforms baseline models when combined with graph neural network (GNN) or large language model (LLM) based graph reasoning models.
Our framework, incorporating grounded knowledge, achieves state-of-the-art performance while providing interpretable evidence.
 \\ \newline \Keywords{Knowledge grounding, Eventuality-centric Knowledge Graphs, Reasoning} }


 
\begin{document}
\newcommand{\yq}[1]{\textcolor{red}{#1}}
\newcommand{\jy}[1]{{\color{blue} #1}}
\newcommand{\remove}[1]{{\color{red} #1}}

\newcommand{\draft}[1]{{\color{blue} #1}}
\newcommand{\methodname}{EventGround}

\maketitleabstract



\section{Introduction}
In the explosion of generative AI in the last couple years, more and more use cases are lending themselves towards large language model (LLM) applicability. Typically an LLM application developer is faced with the question of how to best adapt some particular model for their downstream task; this question usually involves distinguishing among the techniques of few-shot learning or prompt engineering, retrieval augmented generation (RAG) \citep{NEURIPS2020_6b493230}, supervised fine-tuning (SFT), some variant of reinforcement learning for human feedback (RLHF) \citep{schulman2017proximal, bai2022constitutional}, or perhaps some combination of these techniques \citep{balaguer2024rag}.

The motivation for this paper lies in the recognition that despite the impressive capabilities of pre-trained LLMs, they are inherently constrained by the scope and recency of their training data. Internet-scale corpora, which form the foundation of LLM pre-training, are by nature finite snapshots, limited both temporally and in their breadth of covered knowledge. This limitation poses a significant hurdle for applications that necessitate up-to-date information or domain-specific knowledge that postdates the model's training cut-off or falls outside the scope of the training corpus.

Moreover, the dynamic and continuously evolving landscape of human knowledge further complicates the challenge, as new events unfold and specialized domains generate proprietary or niche content that is not publicly accessible. Consequently, developers aiming to deploy LLMs in scenarios that demand current or specialized knowledge must devise strategies for domain adaptation that effectively incorporate this new information into the model.

While RAG \citep{NEURIPS2020_6b493230} provides an ingenious workaround by augmenting model responses with an external knowledge base, this approach sidesteps the core issue of assimilating new knowledge directly into the model itself. Thus, we are compelled to explore alternative methods that enable LLMs to internalize and retain new information through direct training interventions.

Herein lies the crux of our inquiry: how can we construct a training dataset from a body of documents that facilitates the learning of new knowledge through simple SFT techniques? Addressing this question is not only of theoretical interest but also carries significant practical implications for the deployment of LLMs in real-world settings where accuracy, currency, and domain specificity are paramount.

Our investigation into the domain of knowledge ingestion via direct training has yielded several notable contributions:

\begin{enumerate}
\item \textbf{Analysis of Token-based Q\&A Dataset Generation:} We provide a comprehensive evaluation of the standard practice of token-based Q\&A dataset generation. Our findings reveal that this method may not ensure complete or uniform coverage of new knowledge within a document corpus. This observation is critical as it underscores the potential limitations of prevailing dataset preparation strategies and highlights the need for more targeted approaches to knowledge incorporation.


\item \textbf{Development of a Fact-based Generation Process:} In response to the identified shortcomings, we propose a fact-based generation process. This methodology prioritizes the even distribution of attention across all salient facts within a source document, ensuring that each piece of information is adequately represented in the training data. This approach stands to significantly enhance the model's ability to internalize diverse and detailed knowledge from domain-specific corpora.  

\item \textbf{Empirical Validation of SFT for New Knowledge Learning:} We demonstrate that even straightforward SFT can lead to substantial improvements in model performance when handling out-of-domain, post-cutoff knowledge. Our results not only validate the effectiveness of our proposed fact-based generation process but also provide a compelling case for the practicality of SFT as a tool for domain adaptation in LLMs. This contribution has profound implications for the application of generative AI in dynamic fields where staying abreast of the latest information is crucial.  
  
\item \textbf{Benchmarking against Retrieval-Augmented Models:} We extend our study to include a benchmark comparison between our SFT models and those employing RAG. This analysis provides insights into the trade-offs and relative merits of direct training versus retrieval-based augmentation, offering valuable guidance for practitioners in the selection and implementation of knowledge ingestion methodologies.  
  
\item \textbf{Exploration of Hyperparameter Sensitivity:} Recognizing that the tuning of hyperparameters is often a nuanced and impactful aspect of model training, we delve into the sensitivity of our models to a few hyperparameter settings. Our exploration sheds light on the robustness of our findings and sets the stage for future work on optimization strategies tailored to knowledge ingestion tasks.  
\end{enumerate}

Here, we illustrate not only the breadth of our contributions but also their potential implications for the field of AI and LLM application development. By doing so, we aim to provide a thorough understanding of the impact of our research and its relevance to both academic inquiry and practical application.
\section{Related work}
% \section{Background}
% \label{sec:background}

% A narrative is a series of related events that can happen everywhere in our daily life.
Reasoning on narratives is a fundamental task~\cite{mostafazadeh2016corpus,li2018constructing,mori2020finding, jiayang2023storyanalogy} and has attracted much interest in the NLP community. 
% It is related to downstream applications like text summarization and dialogue generation.
The most crucial problem in narrative reasoning is modeling the relationship between events, which often requires background world knowledge~\cite{day1998extensive, mostafazadeh2016corpus}. 
Many large scale knowledge graphs (KGs) such as ATOMIC~\cite{sap2019atomic}, ConceptNet~\cite{speer2017conceptnet}, ASER~\cite{zhang2020aser,zhang2022aser} and GLUCOSE~\cite{mostafazadeh2020glucose} have been constructed in recent years.
% How to leverage the knowledge in these resourceson leverage the knowledge in these resources  remains a problem.
Current solutions on leveraging the knowledge in these resources can be coarsely categorized into the following two groups.
An overview of the two paradigms is presented in Figure~\ref{fig:paradigm}.
% Here the sample question is from Story Cloze Test~\cite{mostafazadeh2016corpus} and the eventuality\footnote{In this work, we use the term ``eventuality'', which includes abstract events, states and activities~\cite{zhang2022aser}, to distinguish from specific events.} subgraph is from ASER.

% \begin{figure*}[t]
% \centering
% \vspace{-10pt}
% \includegraphics[width=1\textwidth]{figs/ParadigmComparison.pdf}
% \vspace{-10pt}
% \caption{Overview of the knowledge model paradigm (left) and the retrieval-and-integration paradigm (right). The knowledge model paradigm pretrains LMs with specially designed objectives, and then further finetunes them to adapt to downstream tasks for prediction. The retrieval-and-integration paradigm retrieves relevant subgraphs of the story context and then makes predictions according to the retrieved subgraphs.}
% \vspace{-10pt}
% \label{fig:paradigm}
% \end{figure*}

% The knowledge model paradigm leverages external KGs by feeding knowledge pieces (usually in the form of triplets) into pretrained language models (PLMs) and fine-tuning the PLMs with carefully designed training objectives~\cite{yu2020cocolm,zhou2021modeling}.
% The fine-tuned knowledge model can then compute a probability based on the narrative sequence to perform inference.
% % However, there is a gap between piecemeal training and inference of the entire sequence.
% % Although the fine-tuned knowledge models can remember the knowledge pieces, it is difficult for the models to understand the entire KG with the graph structure broken down.
% Besides, as a common weakness of PLMs, knowledge models suffer from the poor interpretability where they cannot offer explainable evidence at any steps.

% Knowledge model and Retrieval-and-integration
The knowledge model paradigm leverages external KGs by pretraining LMs with carefully designed objectives.
Most existing knowledge enhanced LMs focused on using entity-centric KGs~\cite{DBLP:conf/acl/ZhangHLJSL19,DBLP:conf/emnlp/PetersNLSJSS19,DBLP:conf/emnlp/FevrySFCK20,DBLP:journals/corr/abs-2007-00849,DBLP:conf/iclr/XiongDWS20,DBLP:journals/corr/abs-1904-09223,DBLP:journals/corr/abs-2107-02137,DBLP:journals/tacl/JoshiCLWZL20}.
As for using external event knowledge, the knowledge model paradigm focus on finetuning language models on event-aware KGs, such as event-pair relation modeling \cite{DBLP:conf/acl/BosselutRSMCC19,DBLP:journals/corr/abs-2110-07178,zhou2021modeling}, whole event recovering/masking  \cite{zhou2022claret, yu2020cocolm}, and correlation-based event ranking \cite{zhou2022eventbert}.
%{\color{gray} entity-knowledge models?}





% Retrieval-and-integration paradigm
The retrieval-and-integration paradigm, in contrast, explicitly retrieves triples or subgraphs from external KGs.
Recent work on reasoning with external KB and texts have explored grounding entities to KGs, such as \cite{sun2018open, sun2019pullnet, xiong2019improving, min2019knowledge, lee2021modeling}, and \cite{lin2019kagnet, feng2020scalable, yasunaga2021qa} in open-domain QA, commonsense QA, and narrative reasoning.
However, most of them ground to entity-centric KGs (e.g. the entity part of ConceptNet \cite{speer2017conceptnet}), which have little or no event knowledge.
Although some \cite{lv2020integrating, lee2019multi, lee2020weakly, li2018constructing} on script reasoning have investigated the usage of events, their methods are restricted to the ``subject-verb-object''-like structured texts in the MCNC task, and have difficulty extending to general free-texts.
In comparison, we tackle the more difficult problem of grounding events in free-texts to eventuality-centric KGs.
The wide adoption of AI critically needs explainability~\cite{hoffman2018metrics}. Thus, despite the appeal of a simpler pipeline (aided by the availability of large LMs), this work extends the retrieval-and-integration paradigm for grounding free-texts to eventuality-centric KGs for narrative reasoning.

 As opposed to event grounding, a similar term ``event linking'' has been used in the literature, where they either focus on cross-document event co-reference \cite{nothman2012event, krause2016event}, or event co-reference to Wikipedia pages \cite{yu2021event}. Moreover, their ``event'' refers to specific happenings such as ``World War II'' rather than the more general eventualities in this work.
% On the other hand, the retrieval-and-integration paradigm leverages external KGs by retrieving knowledge relevant to the input texts and integrating the retrieved knowledge into prediction models.
% With the retrieved knowledge, the prediction models can provide evidence along with the predictions.
% However, early works in the retrieval-and-integration paradigm focus on entity-centric KGs~\cite{zhang2019ernie,liu2020k}, which has small semantic coverage in narrative reasoning tasks. 
% \citet{lv2020integrating} first propose to leverage ASER in the script reasoning task~\cite{li2018constructing}, in which the event texts constructed from event tuples, removing complexities in real-world free texts.
% Even so, the semantic matching algorithm in this work often fails in retrieval or introduces noisy knowledge.


% The wide adoption of AI critically needs explainability~\cite{hoffman2018metrics}. Thus, despite the appeal of a simpler pipeline (aided by the availability of large LMs), this work extends the retrieval-and-integration paradigm for grounding free-texts to eventuality-centric KGs for narrative reasoning.

% We propose an event acquisition pipeline including event extraction, normalization, and abstraction to mine multi-granularity events from input free texts with the help of semantic parsing.
% With the acquired multi-granularity events, we use semantic matching models to ground the events to relevant eventualities in KGs, and thus construct a joint subgraph containing both events from input texts and the eventualities from KGs.
% Then, we encode the subgraph with graph neural networks (GNNs) to make prediction. 
% The constructed subgraph as well as the GNN parameters can provide interpretable evidence.
% Note that we still leverage PLMs for text embedding, but they are constrained in interpretable prediction, thus not used to compute the subgraph probability.
% As a result, our approach doesn't critically rely on large PLMs.








% % Event extraction, normalization, abstraction
% Recent research in extracting events from free-texts. Conceptual abstraction.

% There are two types of approaches on extracting events from natural language texts.
% The first line of work \cite{chambers2009unsupervised, granroth2016happens, li2018constructing, zhang2020aser, zhou2022eventbert, zhou2022claret, zhong2022unsupervised} regards events as verb-rooted sub-trees in syntactic dependencies.
% They first obtain the part-of-speech tagging and dependency parsing information, and search over the text to find matched patterns (e.g., \textit{subject-verb-object}) as events.

% The second line of work \cite{krause2016event, yu2021event}, in contrast, considers the semantic meaning of event components.
% Their event definition follows FrameNet \cite{baker1998berkeley} and ACE \cite{doddington2004automatic}, where each event is composed of a trigger and several arguments. 
% They define patterns over argument roles (e.g., \textit{arg0-verb-arg1}) to extract events.

% {\color{orange} pros and cons, why do we consider the semantic }
% Quality of the extraction results of the syntactic methods highly depends on the design of patterns.
% Further, the events extracted in this way do not contain semantic tags for their arguments.


% \section{Methodology}

\section{\methodname: Grounding free-texts to eventuality-centric knowledge graphs}
\label{sec:method}

% \subsubsection{Event extraction}
% \label{sec:event_extraction}
\begin{figure}[t]
\centering
\includegraphics[width=0.5\textwidth]{figs/pipeline.pdf}
\caption{An overview of \methodname.}
\label{fig:framework}
\end{figure}

% Given a piece of text $S={s_1, s_2, \cdots, s_n}$ with $n$ sentences, we 
% (1) extract events from texts, while preserving the co-reference information of the original text;
% (2) link (or ground) the events to the eventuality-centric KG $\mathcal{G}$; and 
% (3) leverage the grounded commonsense knowledge to facilitate narrative reasoning.

In this section, we present our proposed framework, \methodname.
An overview is presented in Figure \ref{fig:framework}.
To tackle the \textit{event representation} problem, we equip semantic parsing based event extraction ($\S$~\ref{sec:event_extraction}) with an event normalization module ($\S$~\ref{sec:event_normalization}) to separate events from contexts while preserving their arguments' co-reference information.
We solve the \textit{sparsity} problem by with a partial information extraction approach ($\S$~\ref{sec:event_abstraction}).
We empirically prove that these solutions largely alleviate the sparsity problem in $\S$~\ref{sec:ablation}. 
At the end of this section, we discuss grounding the partial events to KGs to obtain joint reasoning subgraphs in $\S$~\ref{sec:event_grounding}, and present both the GNN-based and LLM-based reasoning models in $\S$~\ref{sec:graph_model}. 

% The overall framework of our approach is shown in Figure~\ref{fig:framework}.


\subsection{Obtaining events}
The proposed event acquisition pipeline includes event extraction ($\S$~\ref{sec:event_extraction}), normalization ($\S$~\ref{sec:event_normalization}) and partial information extraction ($\S$~\ref{sec:event_abstraction}).

\subsubsection{Event extraction}
\label{sec:event_extraction}
As shown in the previous example, events do not naturally exist in free texts.
Instead, an event may share arguments with (e.g., \textbf{E1} and \textbf{E2}) or contain another event.
Therefore, a special extraction step is needed to separate events from their contexts.


In this work, we consider the semantic parsing based methods to extract events from their contexts.
For each piece of text $s=[s_1, s_2, \cdots, s_n]$ with $n$ sentences, we conduct semantic role labeling (SRL) on the text to extract a series of verb-centric events $\mathcal{P}=\{p_1, p_2, \cdots, p_m\}$, where each event $p_i=(verb^i, \mathcal{A}^i)$ has a trigger $verb^i$ and a set of arguments $\mathcal{A}^i$.
Each argument $a_j^i\in \mathcal{A}^i$ has  a semantic role $role(a_j^i)\in \{ARG_0, ARG_1, \cdots, ARG_M\}$\footnote{The annotation follows the PropBank \cite{palmer2005proposition} annotation guideline, where the numbered arguments in general correspond to the roles: $ARG_0$-agent; $ARG_1$-patient; $ARG_2$-instrument, benefactive, attribute; $ARG_3$-starting point, benefactive, attribute; $ARG_4$-ending point; $ARG_M$-modifier.}.
In addition, we define the operator $text(p_i)$ to obtain the text of $p_i$.


\subsubsection{Event normalization}
\label{sec:event_normalization}
It is noteworthy that the extracted events suffer from the loss of co-reference information.
For instance, here are three events extracted from a text:\footnote{For simplicity, we do not explicitly show verbs and arguments of the events. All the words in events are lemmatized in our pipeline, which is not shown in the examples.}
\begin{quote}
% \vspace{-6pt}
    (1) The general had some wine at a party. \\
    (2) He felt sleepy. \\
    (3) He said goodbye to them.
% \vspace{-6pt}
\end{quote}
where ``\textit{the general}'' and ``\textit{he}'' refer to the same person, while ``\textit{them}'' refers to another group of people.
A system would not be aware of this co-reference relationship without contexts.
This makes it difficult to reason on the extracted events.

Motivated by previous work \cite{sap2019atomic, fang2021discos} in constructing commonsense KGs, we replace tokens referring to people with special tokens\footnote{Specifically, the spans of personal words are detected by syntactic parsing and animacy classification. We then employ the co-reference information between these spans to normalize all spans that refer to persons.} (e.g., ``\texttt{[P0]},'' ``\texttt{[P0's]},'' ``\texttt{[P1]},'' where different numbers refer to different people).
For instance, ``\textit{the general}'' and ``\textit{he}'' are replaced by ``\texttt{[P0]},'' and ``\textit{them}'' is replaced by ``\texttt{[P1]}.''
Through this normalization process, the co-reference information is preserved:
\begin{quote}
% \vspace{-6pt}
    (1) \texttt{[P0]} had some wine at a party. \\
    (2) \texttt{[P0]} felt sleepy. \\
    (3) \texttt{[P0]} said goodbye to \texttt{[P1]}.
% \vspace{-6pt}
\end{quote}

In addition, the normalization helps reduce event sparsity by removing details in the personal words.
For instance, ``\textit{the general felt sleepy},'' ``\textit{Joe felt sleepy},'' and ``\textit{he felt sleepy}'' will all be normalized to ``\textit{\texttt{[P0]} felt sleepy}.''
This increases their probability of being successfully grounded to KGs.


\subsubsection{Partial information extraction}
\label{sec:event_abstraction}


The normalized events retain rich contextual details from the original texts, which are important for downstream reasoning processes.
However, the sparsity of events can pose challenges in event grounding, especially when most knowledge graphs (KGs) are far from complete \cite{min2013distant, xiong2019improving}. 
For example, a KG is more likely to include a general event like ``\textit{a person is drinking}'' than ``\textit{the general is drinking Sauvignon Blanc on the balcony},'' because the former is more general and likely to occur frequently.

Humans strongly depend on conceptual abstraction to identify similarities among seemingly different concepts and events, which enables generalizations to unfamiliar situations \cite{murphy2004big}.
For instance, we can learn that there is common abstraction between ``\textit{buy a ticket for `Avengers'}'' and ``\textit{buy a ticket for `Harry Potter'},'' and that how the commonality ``\textit{buy a ticket}'' relates to other events such as we should ``\textit{arrive at the theater in time}''.
With this concept in mind, we use a partial information extraction (PIE) phase to obtain partial events as a method of controllable abstraction.
% Research in psychology has shown that the human reasoning process heavily relies on conceptual abstraction \cite{murphy2004big}.
% For instance, we constantly learn to conceptually abstract events.
% This allows us to recognize the commonality between ``\textit{buy a ticket for `Avengers'}'' and ``\textit{buy a ticket for `Harry Potter'}.''
% Moreover, we learn the relationships between  ``\textit{buy a ticket}'' and other events.
% For example, we understand that we should ``\textit{arrive at the theater in time}'' after making the purchase.
% To address the sparsity problem, we apply multi-level abstraction to events. 
% Similar to the abstract thinking process of human beings, we gradually simplify the details of event arguments to obtain partial events.



The partial information extraction is based on the importance of event arguments in semantic role labeling~\cite{palmer2005proposition}.
For instance, $ARG_0$ and $ARG_1$ have the highest importance as they usually specify the subject and objects.
In contrast, the modifier argument $ARG_M$ express the least information, as it usually defines additional constraints of the predicate, such as when and where the event happens.
Specifically, we propose to drop the event arguments in the descending order of their importance.
For event $p=(verb, \mathcal{A})$ with $|\mathcal{A}|=k$, we iteratively drop its argument $a_j\in\mathcal{A}$, such that the roles of dropped arguments follow the order: (1) $ARG_M$\footnote{We do not drop the negation (e.g., \textit{not}, \textit{n't}, \textit{never}) and modals (e.g., \textit{will}, \textit{may}, \textit{can}) modifier arguments, since they are crucial building blocks in discourse as revealed in the linguistics study \cite{jordan1998power}.}, (2) $ARG_2$, $ARG_3$, $ARG_4$, (3) $ARG_1$ and (4) $ARG_0$.
The partial information extraction on event set $\mathcal{P}$ results in a new set of partial events $\mathcal{P}_{abs}$, where $\mathcal{P}_{abs}=\{\hat p_1, \hat p_2, \cdots, \hat p_m\}$.
Each element $\hat p_i=[p_i^0, p_i^1, \cdots]$ is a sequence of partial events correspond to event $p_i\in\mathcal{P}$ ($p_i^0=p_i$).


Below is an example of $\hat p$:
\begin{enumerate}
   \item[$p^0$] ARG0: \underline{\texttt{[P0]}} V: \underline{evacuated} ARG2: \underline{to a relative 's house} ARGM: \underline{last night}. 
    \item[$p^1$] ARG0: \underline{\texttt{[P0]}} V: \underline{evacuated} ARG2: \underline{to a relative 's house}. 
    \item[$p^2$] ARG0: \underline{\texttt{[P0]}} V: \underline{evacuated}.
    \item[$p^3$] V: \underline{evacuated}. 
\end{enumerate}
% \vspace{-5pt}

Each time an argument is dropped, the abstract level of the partial event increases.
Meanwhile, partial events on higher abstract level (e.g. $p^2$, $p^3$) are more likely to have been recorded in KGs, which alleviates the sparsity problem.
In $\S$~\ref{sec:ablation}, we empirically show that the partial information extraction improves the model performance by drastically increasing the hit rate of event grounding.





\subsection{Grounding to eventuality-centric KG} 
\label{sec:event_grounding}

In this section, we discuss the event grounding approach.
In $\S$~\ref{sec:event_matching}, we describe how to map events to eventuality-centric KGs to get the anchor events that have the closest semantic meaning.
In $\S$~\ref{sec:subgraph}, we describe how to retrieve grounded subgraphs based on the anchor events.


\subsubsection{Event matching}
\label{sec:event_matching}

Suppose we have an eventuality-centric KG $\mathcal{G}=(\mathcal{V}, \mathcal{E})$.
$\mathcal{V}$ and $\mathcal{E}$ are the node set and the edge set, respectively.
Each node $v_i \in \mathcal{V}$ is an event with a text attribute $text(v_i)$.
Then, for each event $p\in\mathcal{P}_{abs}$, our goal is to find the node $v\in\mathcal{V}$ (which we term as ``\textit{anchor event}'') that is the most similar to $p$:
\begin{equation}
    v = \arg \min\limits_{v\in\mathcal{V}} d(p, v),
\end{equation}
where $d(\cdot,\space\cdot)$ denotes the distance between events.


To define the similarity, previous work have explored \textit{token-level similarity} by computing the cosine distance for TF-IDF or BM25 vectors \cite{lv2020integrating}.
However, this method overlooks the semantics of events, and constantly fails by mapping to events with high inverse document frequency terms (e.g. ``\textit{\texttt{[P0's]} \underline{lung} gets punched}'' is matched with ``\textit{\texttt{[P0]} has \underline{lung} cancer}'').
Therefore, we turn to use \textit{semantic similarity} to match events.

Specifically, we encode event $p$ and $v$ with sentence transformers \cite{reimers2019sentence},\footnote{\url{https://huggingface.co/sentence-transformers/all-MiniLM-L6-v2}} and compute $d(p,\space v)$ by the L2 distance:

\begin{equation}
    d(p, v) = ||\textrm{SBERT}(text(p)), \textrm{SBERT}(text(v))||_2.
\end{equation}

In practice, not every event can be successfully matched with the correct ones.
We empirically set a threshold $l$ over $d(p, v)$ to filter out the failed matches.\footnote{We sample 100 matching results and empirically set $l$=0.65 that filters out the most failed cases.}
As a result, partial events in $\mathcal{P}_{abs}$ are matched to their anchor events in $\mathcal{G}$, which we denote by $\mathcal{C}$.
$\mathcal{C}=\{\hat c_1, \hat c_2, \cdots, \hat c_m\}$, where each $\hat c_i$ is a sequence of anchor events matched from $\hat p_i$.


\subsubsection{Joint subgraph construction}
\label{sec:subgraph}

\noindent \textbf{Knowledge subgraph retrieval}
Based on the anchor events from the matching results in $\S$~\ref{sec:event_matching}, we aim to retrieve a subgraph $\mathcal{G}_{sub}=(\mathcal{V}_{sub}, \mathcal{E}_{sub})$ from $\mathcal{G}$.
Ideally, $\mathcal{G}_{sub}$ should contain the background world knowledge related to the reasoning, meanwhile cover minimal number of additional eventualities.
Finding such a subgraph is essentially trying to solve an NP-complete Steiner tree problem \cite{garey1977rectilinear, lin2019kagnet}, which is intractable.
As a workaround, we search for the shortest path within $\gamma$-hops between each event pair in $\{(v_a, v_b): v_a\in \hat c_i, v_b\in \hat c_j; \hat c_i, \hat c_j\in \mathcal{C}\}$.
For any path obtained, the nodes and edges along the path are added to $\mathcal{G}_{sub}$.

\noindent \textbf{Joint subgraph construction}
Based on $\mathcal{G}_{sub}$, we construct a joint knowledge enhanced subgraph $\mathcal{G}_{joint}=(\mathcal{V}_{joint}, \mathcal{E}_{joint})$ for reasoning.
Specifically, $\mathcal{G}_{joint}$ includes all the nodes and edges in $\mathcal{G}_{sub}$.
In addition, we add the context events in $\mathcal{P}$ as nodes to $\mathcal{G}_{joint}$, where their grounding relation to anchor events in $\mathcal{C}$ as well as the context relation (between the previous and latter events in the order that they appear in context) are added as edges.

\subsection{Graph reasoning models}
\label{sec:graph_model}

The retrieved subgraphs are then used for reasoning using either a GNN-based reasoning model or an LLM-based reasoning model.

\noindent \textbf{GNN-based reasoning model.}
We first encode the text $s$ and node $v\in\mathcal{V}_{joint}$ using the language model representation:
\begin{equation}
\begin{split}
    \textbf{v} & = f_{\small{\textsc{LM}}}(text(v)), \\
    \textbf{s} & = f_{\small{\textsc{LM}}}(s).
\end{split}
\end{equation}
Then, we employ a GNN module to perform reasoning on the joint subgraph $\mathcal{G}_{joint}$.
We choose the relational graph convolutional networks (RGCN) \cite{schlichtkrull2018modeling} so that the relational information in $\mathcal{G}_{joint}$ can be well modeled.
Specifically, for each layer $l$ in an $L$-layer GNN, the representation $\textbf{h}_i^{(l)}$ of node $i\in\mathcal{V}_{joint}$ is updated by
\begin{equation}
    \mathbf{h}_{i}^{(l+1)} = \sigma \Big(\sum\limits_{r\in\mathcal{R}}\sum\limits_{j\in\mathcal{N}_r(i)}\frac{1}{|\mathcal{N}_r(i)|}\mathbf{W}_r\cdot \mathbf{h}_{j}^{(l)}\Big),
\end{equation}
where $\mathcal{R}$ is the set of edge types in $\mathcal{E}_{joint}$, $\mathcal{N}_r(i)$ denotes the neighborhood with relation $r$ of node $i$, and $\sigma (\cdot)$ is an non-linear activation.
Then, we obtain the vector representation for $\mathcal{G}_{joint}$ by pooling the hidden node embeddings from the last layer
\begin{equation}
    \mathbf{g} = \textrm{Pooling}(\{\mathbf{h}_{i}^{L}: i\in\mathcal{V}_{joint}\}).
\end{equation}
The final prediction comes from 
\begin{equation}
    p(s) \propto \textbf{MLP} (\mathbf{s}+\mathbf{g}),
\end{equation}
where $\textbf{MLP}$ means a multi-layer perceptron module to predict the probability of the output.

\vspace{0.05\linewidth}

\noindent \textbf{LLM-based reasoning model.}
We also explored fusing the eventuality knowledge subgraph $\mathcal{G}_{joint}$ into LLMs.
Since LLMs only receive sequence inputs, we conduct sequentialization on subgraphs in a format similar to \cite{madaan-yang-2021-neural, sakaguchi-etal-2021-proscript-partially}.
Using a transformation function $t(\cdot)$, a subgraph $\mathcal{G}_{joint}$ is transformed into a piece of text $s_{\mathcal{G}_{joint}}$ ($s_{\mathcal{G}_{joint}}=t(\mathcal{G}_{joint})$), which is then fed into LLM as part of the prompts.
We discuss variations of $t(\cdot)$ and other details in $\S$~\ref{sec:exp-setup}.

\section{Experiments}

\subsection{Datasets}

We conduct experiments on three downstream tasks on narrative reasoning.
The statistics are presented in Table \ref{tab:dataset_statistics}.

\noindent$\bullet$ \textbf{Story Cloze Test v1.0} (SCT-v1.0) was proposed by \citet{mostafazadeh2016corpus} to evaluate the understanding of relations between events. 
Given four consecutive sentences, the task is to predict the correct ending from two possible choices.

\noindent$\bullet$ \textbf{Story Cloze Test v1.5} (SCT-v1.5) Later, \citet{sharma2018tackling} introduces a new version to correct the artifacts in the previous release.
For both versions, we follow the common practice \cite{li2019story, yu2020cocolm} to randomly select $100$ validation samples for validation, and use the rest for training.
    
\noindent$\bullet$ \textbf{Multiple Choice Narrative Chain} (MCNC) \cite{granroth2016happens, li2018constructing} is a 5-way multiple choice task that requires a system to predict the ending event given its previous context event sequence.

\begin{table}[t]
\centering
{\small
\begin{tabular}{lccc}
\hline
\textbf{Name} & \textbf{Train} &  \textbf{Valid} & \textbf{Test}\\
\hline
SCT-v1.0 & 1,771 & 100 & 1,871 \\
SCT-v1.5 & 1,471 & 100 & 1,571 \\
MCNC & 140,331 & 10,000 & 10,000 \\
% MPP &  78,528 & 9,816 & 9,817 \\
\hline
\end{tabular}
}
\caption{Statistics of datasets.}
\label{tab:dataset_statistics}
% \vspace{-15pt}
\end{table}


\subsection{Eventuality-centric knowledge graphs}
 
There are eventuality-centric KGs such as ATOMIC \cite{sap2019atomic}, GLUCOSE \cite{mostafazadeh2020glucose} and ASER \cite{zhang2020aser, zhang2022aser}.
In this paper, we conduct experiments on ASER.
The nodes in ASER are eventualities, and the edges between them are the discourse relations (e.g.  ``Precedence'', ``Contrast'' and ``Reason'') defined in Penn Discourse Tree Bank \cite{prasad2008penn}.
To enable grounding normalized events to KGs, we normalize and aggregate eventualities in the ASER-core-100 version\footnote{We obtain the core-100 version by filtering out nodes with frequency lower than 100 from ASER-core: \url{https://hkust-knowcomp.github.io/ASER/}} by detecting and replacing the personal words with aforementioned special tokens.
The resulting normalized ASER graph contains $193k$ nodes and $6.6m$ edges.

\begin{table*}[t]
\centering
{\small
{
\begin{tabular}{ll|ccc}
\hline
\textbf{Method} & \textbf{Size} &  \textbf{SCT-v1.0} & \textbf{SCT-v1.5} & \textbf{MCNC}\\
\hline
\cite{lv2020integrating} & 125M & - & - & 58.66  \\
\cite{zhou2021modeling} & 469M & - & - & 63.62 \\
CoCoLM \cite{yu2020cocolm} & 355M & \underline{97.70} & - & - \\
TransBERT \cite{li2019story} & 355M & 91.80 & 90.30 & - \\
EventBERT \cite{zhou2022eventbert} & 355M & - & \underline{91.33} & 63.50 \\
ClarET \cite{zhou2022claret} & 400M & - & 91.18 & \underline{64.61} \\
\hline
RoBERTa-base \cite{liu2019roberta} & 125M & 92.75\small{$\pm$0.24} & 87.14\small{$\pm$0.39} & 61.28\small{$\pm$0.14}  \\
RoBERTa-large \cite{liu2019roberta} & 355M & 96.74\small{$\pm$0.08} & 92.34\small{$\pm$0.06} & 63.01\small{$\pm$0.12}  \\
DeBERTa-large \cite{he2021debertav3} & 354M & 98.13\small{$\pm$0.34} & 94.67\small{$\pm$0.25}  & 65.67\small{$\pm$0.13}  \\
\hline
\methodname\small{-RoBERTa-base} & 126M &  93.30\small{$\pm$0.11} & 87.65\small{$\pm$0.13} & 62.11\small{$\pm$0.07}   \\
\methodname\small{-RoBERTa-large} & 358M &   97.10\small{$\pm$0.13} & 92.86\small{$\pm$0.05} & 63.96\small{$\pm$0.15}   \\
\methodname\small{-DeBERTa-large} & 358M &   \textbf{98.29\small{$\pm$0.16}} & \textbf{95.01\small{$\pm$0.32}} & \textbf{66.05\small{$\pm$0.12}}  \\
\hline
\end{tabular}
}
}
\caption{Main results on the benchmarks. Numbers are mean and standard deviation of accuracy (\%) over three runs. \underline{Underlined results} are the previous state-of-the-art performance.}
\label{tab:main_result}
\end{table*}



\begin{table}[!t]
\small
\centering
\begin{tabular}{l c c}
\hline
\multicolumn{1}{c}{\textbf{Model}}
& \textbf{SCT-v1.0} & \textbf{SCT-v1.5}\\
\hline
Random                          & 50.00 & 50.00\\
ChatGPT$_\text{Vanilla}$         & 77.80 & 77.00 \\
ChatGPT$_\text{DOT}$ & 67.80 & 69.00 \\
ChatGPT$_\text{Node}$         & 72.00 & \textbf{78.00} \\
ChatGPT$_\text{Node \& Edge}$ & \textbf{79.60} & \textbf{78.00}\\
\hline
\end{tabular}
\caption{ChatGPT evaluation results (accuracy \%). We report the model performance when (1) ChatGPT$_\text{Vanilla}$: no knowledge is provided; (2) ChatGPT$_\text{DOT}$, ChatGPT$_\text{Node}$, and ChatGPT$_\text{Node \& Edge}$: the knowledge subgraphs are transformed into sequences as part of the inputs.}
\label{tab:ChatGPT_Performance}
\end{table}

\subsection{Experimental Setup}
\label{sec:exp-setup}
We implement the event extractor with AllenNLP SRL tools.\footnote{\url{https://github.com/allenai/allennlp}}
To normalize the events, the syntactic parser, animacy classifier, and co-reference tools are from Stanford CoreNLP.\footnote{\url{https://stanfordnlp.github.io/CoreNLP/}} 
In our implementation of the event matching module, due to the large scale of $|\mathcal{V}|$, we employ Faiss \cite{johnson2019billion} to accelerate the similarity search.
% Data hyper-parameters (graph setting)
When retrieving subgraph, we set the shortest path length limit $\gamma$ to 3, meaning that there are at most 2 intermediate nodes between any two anchor nodes along the path.

% Details for the graph reasoning model
We implement the GNN-based reasoning model with Deep Graph Library \cite{wang2019dgl} and Huggingface-Transformers \cite{wolf-etal-2020-transformers}.
% Model hyper-parameters
For finetuning the supervised models, we conduct grid-search over model hyper-parameters. 
The number of convolutional layers $L$ are searched within $\{2, 3, 4\}$, and the hidden size of convolutional layers $\in\{64, 128, 256, 512\}$. 
For relational convolutional layers, the number of bases is searched within $\{-1, 10, 30\}$.
We use the Adam \cite{kingma2015adam} optimizer with cosine learning rate schedule to optimize the models.
The learning rate is set to $1e-5$ for all the ``base'' models, and $5e-6$ for all the ``large'' models.
All the experiments are run on 4 NVIDIA Tesla-V100 GPUs. 


For the LLM-based reasoning model, we adopt ChatGPT~\cite{openai2022chatgpt}.~\footnote{The evaluation is performed in September 2023.}
We consider three implementations for the graph sequentialization function $t(\cdot)$: (1, DOT) using the DOT language to represent graphs \cite{gansner1993technique, madaan-yang-2021-neural, sakaguchi-etal-2021-proscript-partially}; (2, Node \& Edge) instead of using node indexing as in DOT, we try directly inputing all the nodes and edges (e.g., ``\texttt{[P0] buy a boat --> [P0's] nearby marina have a race; [P2] prepare --> [P2] go to sleep; ...}''); (3, Node) only the nodes are fed into ChatGPT (e.g., ``\texttt{[P0] buy a boat; [P0's] nearby marina have a race ...}'').
The prompt template is:
``\texttt{Event knowledge on narrative choice A: \{$t(\mathcal{G}_{joint, A})$\} \textbackslash n Event knowledge on narrative choice B: \{$t(\mathcal{G}_{joint, B})$\} \textbackslash n Question:\{\} \textbackslash n Answer:''}.
As a baseline, we also test ChatGPT without the additional knowledge (denoted by ``ChatGPT$_\text{Vanilla}$'').
For SCT-v1.0, we report results on its test set (sampled 500 instances). 
Since the test set of SCT-v1.5 is no longer publicly available\footnote{\url{https://competitions.codalab.org/competitions/15333}} at the time we ran this experiment, we report the results on its validation set.
We do not report the performance on MCNC because the lengths of most instances in this set exceed the maximum input length.




\subsection{Main results}

The main results on the three datasets are presented in Table \ref{tab:main_result} and \ref{tab:ChatGPT_Performance}. 
Per-task performance comparisons are presented in Appendix \ref{sec:appendix_detail_results}.

As shown in Table \ref{tab:main_result}, when coupled with a GNN-based reasoning model, our proposed framework achieves consistent performance gain over different backbone models.
Moreover, compared with existing knowledge enhanced models, we achieve SOTA performance in three narrative reasoning tasks.
The knowledge also benefits our LLM-based reasoning model (Table~\ref{tab:ChatGPT_Performance}), especially when the subgraphs are transformed using the ``$\text{Node \& Edge}$'' setting.



\subsection{Ablation study}
\label{sec:ablation}
We conduct ablation studies to investigate the contribution of each component in our framework.

\begin{table}[t]
\centering
{\small
\begin{tabular}{lcc}
\hline
     & \methodname\tiny{-RB} &  \methodname\tiny{-BB} \\
\hline
w/o know. & 92.75\small{$\pm$0.24} & 83.63\small{$\pm$1.16} \\
w/o extract. & 91.86\small{$\pm$0.21} & 83.74\small{$\pm$0.38} \\
w/o norm.   & 92.43\small{$\pm$0.46} & 83.98\small{$\pm$0.87} \\
\hline
w/o PIE & 92.81\small{$\pm$0.32} & 83.88\small{$\pm$1.40} \\
- ARGM & 93.17\small{$\pm$0.25} & 84.79\small{$\pm$1.37} \\
- ARG2,3,4 & 93.03\small{$\pm$0.49} & 84.53\small{$\pm$0.60} \\
- ARG1 & \textbf{93.30\small{$\pm$0.11}} & \textbf{85.78\small{$\pm$0.74}} \\
\hline
\end{tabular}
}
\caption{Effect of event extraction, normalization and partial information extraction (PIE). The mean and standard deviation of accuracies on SCT-v1.0 are reported, where ``RB'' and ``BB'' refer to RoBERTa-base and BERT-base versions.}
% \vspace{-15pt}
\label{tab:main_ablation}
\end{table}

\subsubsection{Effect of event extraction, normalization, and partial information extraction}
As shown in Table \ref{tab:main_ablation}, we ablate the event extraction (``w/o extract.''), the event normalization (``w/o norm.'') and the partial information extraction (``w/o PIE'' and ``- ARGX'') respectively.
Specifically, when ablating the event extraction module, we instead use the whole sentence for event grounding.
When ablating the event normalization part, we skip the normalization step, and use the raw events for grounding.
For partial information extraction, we drop event arguments in the order described in $\S$~\ref{sec:event_abstraction}, where the highest level (``- ARG1'') contains all the partial events in the previous levels.
The baseline (``w/o know.'') shows the results of vanilla language models, which do not leverage any external knowledge.

We have several observations. 
First, the event extraction and normalization steps are necessary. 
When removed, the performance relative to the baseline does not improve, or even drops.
Second, the partial information extraction step is crucial.
By only taking the first level of partial events (removing modifier arguments), we have seen considerable performance gain.
The model reaches its best performance after dropping ARG1.

In $\S$~\ref{sec:method}, we discuss the \textit{sparsity} of events.
Here, we conduct both automatic and human evaluation to discuss how our method contribute to the alleviation of sparsity.

\noindent$\bullet$ \textbf{Automatic Evaluation} (Figure \ref{fig:ablation_matching}) We analyze by automatic measures: (1) the average L2 distance $\bar d$ in event matching ($\S$~\ref{sec:event_matching}), and (2) the percentage of events considered as successful match, i.e. with L2 distance below $l=0.65$ (hit rate). 


\noindent$\bullet$ \textbf{Human Evaluation} (Table \ref{tab:human_eval}, Figure \ref{fig:ablation_f1}) 
    We evaluate the matching results by human annotation.
    Three domain experts are asked to annotate whether event matching is successful for 50 stories ($\sim$500 events) randomly sampled from the validation set of SCT v1.0. 
    The Fleiss's Kappa value is $0.7414$.
    We obtain ground-truth labels by majority vote, and present the accuracy of different event matching methods in Table \ref{tab:human_eval}.
    To investigate the effect of the threshold $l$ used in $\S$~\ref{sec:event_matching}, we visualize F1 scores under different threshold values in Figure \ref{fig:ablation_f1}. 
    

\begin{figure}[ht]
% \vspace{-10pt}
\centering
\includegraphics[width=0.5\textwidth]{figs/norm_abs.png}
\caption{A comparison on the event grounding performance under different settings. The bar plot (with $y$-axis on the left) shows the percentage hit rate of event matching. The lines show the average L2 distance $\bar d$. 
We do not conduct normalization for ``w/o extract.''. }
\label{fig:ablation_matching}
% \vspace{-10pt}
\end{figure}

\begin{table}[ht]
\centering
{\small
\begin{tabular}{lcc}
\hline
     & w/o norm. &  w/ norm. \\
\hline
w/o extract. & 4.7 & - \\
\hline
w/o PIE & 7.5 & 37.5 \\
- ARGM & 10.0 & 56.2 \\
- ARG2,3,4 & 14.6 & 73.4 \\
- ARG1 & 9.9 & 86.6 \\
\hline
\end{tabular}
}
\caption{Human evaluation for the accuracy of event matching (\%). }
% \vspace{-10pt}
% \vspace{-10pt}
\label{tab:human_eval}
\end{table}

We can observe that: 
1) Directly matching sentences to KGs (w/o extract.) has rather low performance, which necessitates the event extraction stage.
2) The event normalization step drastically improves the matching performance. 
Removing normalization step can decrease the accuracy by up to $76.7\%$.
3) In general, the matching performance gradually increases as the abstract level increases.
4) The Pearson's $r$ between automatic and human evaluation results is $0.8977$, indicating thresholding on $L2$ distance is a reasonable way to automatically filter out poorly matched events.  
Moreover, from Figure \ref{fig:ablation_f1}, we learn that event extraction, normalization, and partial information extraction improve not only performance but also robustness of event matching. 
Notably, our main model (w/ norm. -ARG1) has much higher success rate than the other models, and it is meanwhile insensitive to the tuning of threshold $l$.




% Various threshold-performance curves
\begin{figure}[ht]
% \vspace{-10pt}
\centering
\includegraphics[width=0.5\textwidth]{figs/F1-threshold.png}
\caption{The F1-score to threshold curves. They reflect the event matching performance under different threshold $l$. }
\label{fig:ablation_f1}
% \vspace{10pt}
\end{figure}








\subsubsection{Effect of model structure}

We test the GNN-based reasoning model performance with different backbone text encoders (Table~\ref{tab:ablation_text_encoders}).
Compared with the baselines (``w/o know.''), our framework consistently improves performance across different versions of LMs.

We also investigate the effect of different GNN configurations in Table \ref{tab:ablation_gnn}.
Apart from the relational convolutional layers (RGCN \cite{schlichtkrull2018modeling}), we additionally test GIN \cite{xu2018powerful} and GCN \cite{kipf2016semi}, which do not model the edge type information.
We can observe that RGCN outperforms GIN and GCN under the same settings.
This indicates the discourse relation knowledge in ASER is beneficial for narrative reasoning.

We evaluate the LLM-based reasoning model under different graph sequentialization settings (Table~\ref{tab:ChatGPT_Performance}).
It is noteworthy that ChatGPT faces difficulties in understanding the knowledge represented in DOT language, resulting in a performance drop of approximately 10\%. 
One possible reason for this is that the model was not trained to comprehend such structured representations. 
Additionally, providing only node information to the model does not yield significant benefits. 
The model demonstrates improved performance when using the "Node \& Edge" representation of graphs.

\begin{table}[t]
\centering
{\small
\begin{tabular}{lccc}
\hline
\textbf{Model} & \textbf{Type} &  \textbf{w/o know.} & \textbf{w/ know.}\\
\hline
\multirow{2}{*}{BERT} &
    \multirow{1}{*}{base} & 83.63\small{$\pm$1.16} & 85.78\small{$\pm$0.74} \\
    & \multirow{1}{*}{large} & 88.85\small{$\pm$0.23} & 90.49\small{$\pm$0.41} \\
\hline
\multirow{2}{*}{RoBERTa} &
    \multirow{1}{*}{base} & 92.75\small{$\pm$0.24} & 93.30\small{$\pm$0.11} \\
    & \multirow{1}{*}{large} & 96.74\small{$\pm$0.08} & 97.10\small{$\pm$0.13} \\
\hline
\multirow{2}{*}{DeBERTa} &
    \multirow{1}{*}{base} & 96.03\small{$\pm$0.17} & 96.38\small{$\pm$0.14} \\
    & \multirow{1}{*}{large} & 98.13\small{$\pm$0.24} & 98.29\small{$\pm$0.16} \\
\hline
\end{tabular}
}
\caption{Effect of different text encoders. Three backbone language models BERT \cite{devlin2018bert}, RoBERTa \cite{liu2019roberta}, and DeBERTa \cite{he2021debertav3} are tested on SCT-v1.0.}
\label{tab:ablation_text_encoders}
% \vspace{-20pt}
\end{table}


\begin{table}[t]
\centering
{\small
\begin{tabular}{cc|cc}
\hline
 & & \multicolumn{2}{c}{$L$-layer} \\
n-hidden & conv. & 2 &  3\\
\hline
\multirow{3}{*}{128} &
    \multirow{1}{*}{\small{RGCN}} & 93.30\small{$\pm$0.11} & 92.97\small{$\pm$0.17} \\
    & \multirow{1}{*}{\small{GIN}} & 92.93\small{$\pm$0.37} & 92.57\small{$\pm$0.24} \\
    & \multirow{1}{*}{\small{GCN}} & 92.95\small{$\pm$0.10} & 93.16\small{$\pm$0.22} \\
\hline
\multirow{3}{*}{256} &
    \multirow{1}{*}{\small{RGCN}} & 93.14\small{$\pm$0.20} & 93.12\small{$\pm$0.17} \\
    & \multirow{1}{*}{\small{GIN}} & 93.05\small{$\pm$0.42} & 92.41\small{$\pm$0.31} \\
    & \multirow{1}{*}{\small{GCN}} &  92.94\small{$\pm$0.13} & 92.86\small{$\pm$0.21} \\
\hline
\end{tabular}
}
\caption{Effect of different GNN settings on SCT-v1.0.  }
\label{tab:ablation_gnn}
% \vspace{-15pt}
\end{table}




\subsection{Case study}
\label{sec:case}
A running example is presented in Figure \ref{fig:case_study}.
The top three nodes that our model focuses on are ``\texttt{[P0]} study,'' ``\texttt{[P0]} pass the test,'' and ``\texttt{[P0]} believe.''
They are highly related to the correct candidate ending 1.
Also note the path (``\texttt{[P0]} study,'' \textit{Reason}, ``it go well,'' \textit{Conjunction},  ``\texttt{[P0]} pass the test'') could be explained as the causal story:
Someone studies hard, so it (the learning, or the exam) goes well, and he/she passes the test.

\begin{figure}[!h]
\centering
\includegraphics[width=0.5\textwidth]{figs/case_study.pdf}
\caption{An example from SCT-v1.0. The top-10 node attention weights are shown in the barplot. The top-3 nodes are \underline{\textbf{bolded and underlined}} .}
\label{fig:case_study}
\end{figure}



\section{Conclusion}
\vspace{-8pt}

We point out two critical problems on grounding free-texts to eventuality-centric KGs, namely the \textit{event representation} and \textit{event sparsity} problems.
We propose a simple while effective approach, \methodname, to address these problems and to leverage the retrieved graph knowledge for narrative reasoning.
Empirical results demonstrate its consistent performance improvement.
% Further investigation reveals that its normalization and abstraction components drastically alleviates event sparsity.
Further investigation reveals that the normalization and partial information extraction components  drastically improve the grounding performance by alleviating event sparsity.


\newpage
\section*{Limitations}
In event normalization, we only normalize personal words in event as it is the most common spans that worth normalization, normalization of other type of information are not considered, which we leave for future work.
When grounding to event-centric KGs, we consider finding the shortest paths to retrieve the knowledge subgraph due to high computational complexity of solving the Steiner tree problem.
Other retrieval methods (e.g. reinforcement learning based) could also be considered.



\section*{Acknowledgements}
The authors of this paper were supported by the NSFC Fund (U20B2053) from the NSFC of China, the RIF (R6020-19 and R6021-20) and the GRF (16211520 and 16205322) from RGC of Hong Kong. We also thank the support from the UGC Research Matching Grants (RMGS20EG01-D, RMGS20CR11, RMGS20CR12, RMGS20EG19, RMGS20EG21, RMGS23CR05, RMGS23EG08).

% Entries for the entire Anthology, followed by custom entries
\bibliography{anthology,custom}
\bibliographystyle{acl_natbib}

\clearpage

\section{Properties of the dominance}

\PropDomLinked*

\begin{proof}
  We need to show that the intrinsic order on \(\Sigma\) coincides with the path relation. First, we observe that \(\bot   \specle   \top\); indeed, fixing a map \(f :  \Sigma   \to   \Sigma\), by Phoa's principle, evaluation at boundary obtains a pair \((f~ \bot , f~ \top )\) such that \(f~ \bot\) implies \(f~ \top\). In one direction, suppose \(x  \pathle  y\), which means we have a path \(l :  \Sigma   \to   \Sigma\) whose boundary is determined by \(x, y\). Fixing \(f :  \Sigma   \to   \Sigma\), we want to show that \(f~x  \to  f~y\). But this follows by evaluating \(\bot   \specle   \top\) at the $\Sigma$-predicate \(f  \circ  l :  \Sigma   \to   \Sigma\). Conversely, if \(x \specle  y\), then we have in particular \(x  \to  y\), which by Phoa's principle uniquely determines a path \(l :  \Sigma   \to   \Sigma\).
\end{proof}

\section{Properties of predomains}

\PropPredomainProperties*
\begin{proof}
  Replete types are both complete and boundary separated because the latter can be defined by orthogonality conditions and is satisfied by $\Sigma$: it is complete by the axioms of SDT (\cref{def:axioms}) and it is boundary separated by \cref{prop:sigma-boundary-sep}. Assuming Phoa's principle, the proof that replete types are linked can be found in Taylor~\cite[Corollary 2.10]{taylor:1991} and Reus~\cite[Corollary 6.1.16]{reus:1995}. Lastly, one can show that the objects whose link relation is antisymmetric can be defined as an orthogonality condition, thus we know that the link relation on every replete type is antisymmetric, hence it follows the intrinsic order on replete types are also partial orders since they are linked.
\end{proof}

\PropSealingSigDecidable*

\begin{proof}
  We observe that $\P \vee A$ is defined as the pushout of the projections of $A \times \P$ as indicated below: 
  \[
  \DiagramSquare{
      ne = A, 
      se = \P \vee A,
      nw = A \times \P,
      se/style = pushout,
      sw = \P, 
      east = \eta_{\P \vee -}, 
      south = \star,
      }
  \]
  Using the fact that $A$ has $\Sigma$-equality, we obtain a map $f : (\P \vee A) \times (\P \vee A) \to \P \vee \Sigma$ such that $f(\eta_{\P \vee -}(x), \eta_{\P \vee -}(y)) = \eta_{\P \vee -}(x = y)$ and $f(\star, -) = f(-, \star) = \star$. The desired characteristic map can then be defined as $\sigma \circ f$, where $\sigma : \P \vee \Sigma \to \Sigma$ is defined as follows:
  \begin{align*}
    &\sigma : \P \vee \Sigma \to \Sigma\\ 
    &\sigma(\eta_{\P \vee -}(\phi)) = \P \vee \phi\\ 
    &\sigma(\star) = \top
  \end{align*}  
  For any $u, v : \P \vee A$, if $u = v = \eta_{\P \vee -}(x)$ for some $x : A$, then we have $\sigma(f(u, v)) = \sigma(\eta_{\P \vee -}(x = x)) = \P \vee (x = x) = \top$. Otherwise, we have that $u$ or $v$ is $\star$, which means $\P$ holds and so $\sigma(f(u, v)) = \sigma(\star) = \top$ as well. Conversely, suppose $\sigma(f(u, v)) = \top$, and that $u = \eta_{\P \vee -}(x)$ and $v = \eta_{\P \vee -}(y)$, which means that $\sigma(\eta_{\P \vee -}(x = y)) = \P \vee (x = y)$ holds. If $\P$ holds, we are done as $(\P \vee A) \cong 1$ in this case. Otherwise, we have $x = y$, and so $u = \eta_{\P \vee -}(x) = \eta_{\P \vee -}(y) = v$. Lastly, if either $u$ or $v$ is the unique element $\star$ then we may discharge the case as above. 
\end{proof}

\PropFuncPointwise*

\begin{proof}
  This is Proposition 5.4.4 of Phoa~\cite{phoa:1991}. We just show the case for the function types. Given a path $f \le_{X \to B} g$, it is clear that we may construct a path $f~x \le B g~x$ for all $x : X$. Conversely, suppose we are given a path $f~x \le B g~x$ for all $x : X$. By \cref{prop:predomain-properties}, $B$ is boundary separated, and so such paths are necessarily unique, and so we have a function $\alpha : X \to \Sigma \to B$ such that $\alpha(x)$ is a path $f~x \le B g~x$. We then obtain a path $f \le_{X \to B} g$ by taking the exponential transpose of $\alpha$.  
\end{proof}

\PropLiftPointwise*

\begin{proof}
  In the forward direction, we have a path \(x{ \downarrow }\) \(\sqsubseteq\) \(y{ \downarrow }\), which means \(x{ \downarrow }\) implies \(y{ \downarrow }\) as the \(\Sigma\) is linked by \cref{prop:dom-linked}. Suppose \(x { \downarrow }\), and let \(f : A  \to   \Sigma\) be arbitrary. By assumption, we have that \(f'(x)  \to  f'(y)\), where \(f'(( \phi , u)) =  \phi   \mathbin{\angle}  f  \circ  u\). In other words, we have \(x { \downarrow }   \mathbin{\angle}  f(x)\) implies \(y { \downarrow }   \mathbin{\angle}  f(y)\). Since \(x { \downarrow }\) holds, we have that \(f(x)  \to  f(y)\), which by definition means \(x  \sqsubseteq _A y\). 
    
  In the backward direction, let \(\alpha  :  x { \downarrow }   \to  x  \sqsubseteq _A y\) be the given partial path. We may define a total path \(\beta  :  \Sigma   \to   \mathsf{L} A\) between \(x\) and \(y\) by setting $\beta ( \phi ) = ( x { \downarrow } ,  \lambda  p.~ \alpha ~p~ \phi )$. Thus we have $x \le_{\lift{A}} y$ as required.
\end{proof}

\PropLub* 

\begin{proof}
  The final \(\mathsf{L}\)-coalgebra is equipped with a global element \(\infty  :  \overline{\omega}\) that can be thought of as the ``point at infinity''. Define \(f_ \infty\) be the element determined by the unique extension \(\overline{f}  :  \overline{\omega}   \to  A\) evaluated at the invariant point \(\infty\). 
  \begin{enumerate}
    \item First we show that \(f_ \infty\) is an upper bound for \(f\). Fixing \(i:  \omega\), we need to show that \(f~i  \sqsubseteq ^ \circ _A f_ \infty\). Because \(\overline{f}\) extends \(f\), it suffices to show \(\overline{f} ~i  \sqsubseteq ^ \circ _A f_ \infty\). Using the fact that every map is monotone with respect to the specialization order, the result holds because \(i  \sqsubseteq ^ \circ _{ \overline{\omega} }  \infty\).
    
    \item Let \(\alpha\) be an upper bound for \(f\). We need to show that \(f_ \infty   \sqsubseteq ^ \circ   \alpha\). If the principal lower set \({ \downarrow }( \alpha )\) is complete, we have the following lifting situation:
    \[\begin{tikzpicture}[diagram]
      \path
      (-1.5,0) node (I) {$\omega$}
      (1.5,0) node (F) {$\overline{\omega}$}
      (0,-2) node (A) [align=center] {${\downarrow}(\alpha) = \{ a \mid a \specle \alpha \}$};
      \draw[embedding] (I) to (F); 
      \draw[->] (I) to node[left] {$f$} (A); 
      \draw[->,exists] (F) to node[right, shift={(0.1,0)}] {$\tilde{f}$} (A);
      \end{tikzpicture}
      \]
      
      In the above \(\tilde{f}\) is the unique extension of \(f\) considered as a map \(\omega   \to  { \downarrow }( \alpha )\). By uniqueness of \(\overline{f}\) as the extension of \(f :  \omega   \to  A\), \(\tilde{f}\) is equal to \(\overline{f}\) considered as maps \(\overline{\omega}   \to  A\). Consequently we have that \(f_ \infty  =  \overline{f} ( \infty ) =  \tilde{f} ( \infty )\), so the result follows by observing that \(\tilde{f} ( \infty )  \in  { \downarrow }( \alpha )\).
    
      It remains to show that \({ \downarrow }( \alpha )\) is complete. We can express the principal lower set as follows:
      \[\begin{aligned}           { \downarrow }( \alpha ) &=  { \left \{ a  \mid  a  \sqsubseteq ^ \circ   \alpha \right \} } \\  
          &=  { \left \{ a  \mid   \forall  f:A \to \Sigma ~ f(a)  \to  f( \alpha ) \right \} } \\  
          &=  \bigcap _{f:A \to \Sigma }  { \left \{ a  \mid  f(a)  \to  f( \alpha ) \right \} }        \end{aligned}\] 
          
      Because complete types are internally complete, the result would follow if we can show that \(S =  { \left \{ a  \mid  f(a)  \to  f( \alpha ) \right \} }\) is complete. We may show that \(S\) can be computed as follows: 
      \[
      \DiagramSquare{
      ne = A, 
      se = \Sigma \times \Sigma,
      nw = S,
      nw/style = pullback,
      sw = \Sigma^\Sigma,
      east = {\langle f, f(\alpha)\rangle}, 
      south = \partial,
      }
      \]
      Since $S$ can be defined as the limit of a diagram of complete types, it is complete as well. 
\end{enumerate}
\end{proof}

\PropMonotone*

\begin{proof}
  Given $a \le b$, we derive a path $\Sigma \to B$ whose boundary is $(f~a, f~b)$ by postcomposing with $f$, and so $f~a \le f~b$ as well. 
\end{proof}

\PropCont* 

\begin{proof}
  Predomains are complete, so we have the following extensions of $d$ and $f \circ d$: 
  \[
  \begin{tikzpicture}[diagram]
    \SpliceDiagramSquare<sq/>{
    width = 3cm,
    nw = \omega, 
    sw = \overline{\omega},
    ne = A, 
    se = B,
    east = f, 
    south = \overline{f \circ d}, 
    west/style = embedding, 
    south/style = {->,exists},
   } 
   \draw[->,exists] (sq/sw) to node[desc] {$\overline{d}$} (sq/ne);
  \end{tikzpicture}
  \]
  Because extensions along $\omega \hookrightarrow \overline{\omega}$ are unique for complete types, we have $f \circ \overline{d} = \overline{f \circ d}$. But by definition of the synthetic $\omega$-join, this means that $f(\sup d) = f(\overline{d}(\infty)) = \overline{f \circ d}(\infty) = \sup(f \circ d)$. 
\end{proof}

\section{Domains and admissibility}

\begin{proposition}\label{prop:path-implies-intrinsic}
  The path relation need not coincide with the intrinsic order, but a path $\alpha : x \pathle y$ always implies $x \specle y$. 
\end{proposition}

\begin{proof}
  Suppose we have a path \(l :  \Sigma   \to   \Sigma\) whose boundary is determined by \(x, y\). Fixing \(f :  \Sigma   \to   \Sigma\), we want to show that \(f~x  \to  f~y\). But this follows by evaluating \(\bot   \specle   \top\) at the $\Sigma$-predicate \(f  \circ  l :  \Sigma   \to   \Sigma\). 
\end{proof}

\begin{proposition}\label{prop:invariant-top}
The invariant point $\infty$ is the top element of $\overline{\omega}$ with respect to the intrinsic preorder. 
\end{proposition}

\begin{proof}
  Observe that for any $i : \overline{\omega}$ we have a path $\alpha : i \pathle \infty$ defined by $\alpha(\phi, n) = \phi \vee i(n)$. Thus we have $i \specle \infty$ by \cref{prop:path-implies-intrinsic}. 
\end{proof}

\BugAdmissible*

\begin{proof}
  Suppose $S \subseteq A$ is downward closed and closed under synthetic $\omega$-joins. We must complete the following lifting problem: 
  \[
  \Local{\omega}{\overline{\omega}}{S}[f][?]
  \]
  Because $S$ is complete, we have the following extension of $\omega \xrightarrow{f} A \xhookrightarrow{\iota} S$:
  \[
  \begin{tikzpicture}[diagram]
    \SpliceDiagramSquare<sq/>{
      nw = \omega,
      sw = S,
      ne = \overline{\omega},
      south/style = {opacity = 0},
      east/style = {opacity = 0},
      north/style = >->,
      west = f,
    }
    \node[below = of sq/sw] (S) {$A$};
    \draw[->,dotted] (sq/ne) to node[right] {$\overline{\iota f}$} (S);
    \draw[->,dotted] (sq/ne) to node[right] {$?$} (sq/sw);
    \draw[>->] (sq/sw) to node {} (S);
  \end{tikzpicture}
  \]
  Thus it suffices to show that $\overline{\iota f}$ factors through $S$. Fixing $i : \overline{\omega}$, we want to show that $\overline{\iota f}(i) \in S$. By the assumptions on $S$, we have that $\sup (\iota f) = \overline{\iota f}(\infty)$ is in $S$ and that $a \in S$ whenever $a \le \overline{\iota f}(\infty)$. Thus it suffices to show $\overline{\iota f}(i) \le \overline{\iota f}(\infty)$, which follows from \cref{prop:invariant-top}. 
\end{proof}

\begin{proposition}
  The intersection of a family of admissible subsets of a domain is admissible.
\end{proposition}

\begin{proof}
  The least element is contained in the intersection as it is contained in every fiber. Suppose that \(f :  \omega   \to  A\) is a synthetic \(\omega\)-chain such that \(f_i  \in   \bigcap  F\). But since \(f_i  \in  F_j\) for every \(j : J\),  we have \(\bigvee  f  \in  F_j\) as every \(F_j\) is admissible, which means \(\bigvee  f  \in   \bigcap  F\) as well.
\end{proof}

\begin{proposition}
  If \(P, Q\) are \(\Sigma\)-subsets of a domain \(A\), then the exponential subobject \(Q^P\) is an admissible subset of \(A\). 
\end{proposition}

\begin{proof}
  Let \(f_i  \in  Q^P\) for some synthetic \(\omega\)-chain \(f\). Suppose that \(\bigvee  f  \in  P\). We need to show that \(\bigvee  f  \in  Q\). By the universal property of $\bigvee$, we may assume that \(f_i  \in  P\) for some \(i :  \omega\). By assumption, this means that \(f_i  \in  Q\) and thence \(\bigvee  f  \in  Q\) as every \(\Sigma\)-subset is monotone in the synthetic order.
\end{proof}

\section{Properties of the denotational semantics} 

\PropCostAlgLaws*

\begin{proof}
  Routine computation using the distributive law $\tau$ of \(\mathbb{C}   \times  -\) over the lifting monad.
\end{proof}

\section{Properties of the computational semantics}

\PropEvalFunc*

\begin{proof}
  Consider the following subset:  \[\begin{aligned}     P =  { \left \{ \alpha   \mid   \forall   { \left [ e, v, v' \right ]} ~   \alpha (e, v) { \downarrow }   \land   \alpha (e, v') { \downarrow }   \to  v = v' \right \} }    \end{aligned}\]  
  
  We may check that \(P\) is admissible by means of \cref{cor:admissible} and proceed by fixed-point induction. Suppose that \(\alpha   \in  P\) and that \(\Phi _ \mathsf{eval} ( \alpha )(e, v) { \downarrow }\) and \(\Phi _ \mathsf{eval} ( \alpha )(e, v') { \downarrow }\). We need to show that \(v = v'\). We proceed by cases on \(\mathsf{out} (e)\).
  \begin{enumerate}
    \item{If \(e \mapsto  c',e'\), we may deduce that \(\alpha (e',v) { \downarrow }\) and \(\alpha (e',v') { \downarrow }\), so the result follows from the assumption that \(\alpha   \in  P\). }    \item{Otherwise, we have that \(e = v\) and \(e = v'\) by definition of \(\Phi _ \mathsf{eval}\), and so \(v = v'\).} 
  \end{enumerate}
\end{proof}

\PropEvalSeq* 

\begin{proof}
  Consider the following subset of \(\Pi _{A :  \mathsf{tp} }.~ \mathsf{U} \mathsf{F} A   \to   \mathsf{U} \mathsf{F} A   \to   \mathsf{T} (1)\):
  \[\begin{aligned}     P =  { \left \{ \alpha   \mid   \forall   { \left [ e \right ]} ~   \alpha (e, v) { \downarrow }   \land   \mathsf{eval} (g~v,  \mathsf{ret} (w)) { \downarrow }   \to   \mathsf{eval} (e; g,  \mathsf{ret} (w)) =  \alpha (e, v) +  \mathsf{eval} (g~v,  \mathsf{ret} (w)) \right \} }    \end{aligned}\]  \par{It suffices to show that \(\mathsf{eval}   \in  P\). Observing that \(P\) is admissible, we proceed by fixed-point induction. Suppose that \(\alpha   \in  P\), \(\Phi _ \mathsf{eval} ( \alpha )(e, v) { \downarrow }\), and that \(\mathsf{eval} (g~v,  \mathsf{ret} (w)) { \downarrow }\). We need to show that \(\mathsf{eval} (e; g,  \mathsf{ret} (w)) =  \Phi _ \mathsf{eval} ( \alpha )(e, v) +  \mathsf{eval} (g~v,  \mathsf{ret} (w))\). We proceed by cases on \(\mathsf{out} (e)\). }  \begin{enumerate}
    \item{      If \(e  \mapsto  c',e'\), then we compute: 
      \[\begin{aligned}          \mathsf{eval} (e; g,  \mathsf{ret} (w)) &= c'  \boxplus   \mathsf{eval} (e'; g,  \mathsf{ret} (w)) \\         &= c' +  \alpha (e', v) +  \mathsf{eval} (g~v,  \mathsf{ret} (w)) \\         &=  \Phi _ \mathsf{eval} ( \alpha )(e, v) +  \mathsf{eval} (g~v,  \mathsf{ret} (w)) 
       \end{aligned}\]      \par{Where the first equality follows from the assumption that \(\alpha   \in  P\) and the second by the definition of \(\Phi _ \mathsf{eval}\). }    }    \item{Otherwise, we have that \(e~ \mathsf{val}\). Since \(\Phi _ \mathsf{eval} ( \alpha )(e,  \mathsf{ret} (v)) { \downarrow }  =  (e =  \mathsf{ret} (v), 0) { \downarrow }  = (e =  \mathsf{ret} (v))\) holds, we can compute: }    \[\begin{aligned}       \mathsf{eval} (e; g,  \mathsf{ret} (w)) &=  \mathsf{eval} ( \mathsf{ret} (v); g,  \mathsf{ret} (w)) \\      &=  \Phi _ \mathsf{eval} ( \mathsf{eval} )( \mathsf{ret} (v); g,  \mathsf{ret} (w)) \\      &=  \mathsf{eval} (g~v,  \mathsf{ret} (w))     \end{aligned}\]    \par{But this is what we needed to show since \(\Phi _ \mathsf{eval} ( \alpha )(e,  \mathsf{ret} (v)) = 0\). }  
\end{enumerate}
\end{proof}

\PropCompIndSeq* 

\begin{proof}
  Consider the subset \(P  \subseteq   \mathsf{U} \mathsf{F} A   \to   \mathsf{U} \mathsf{F} A   \to   \mathsf{T} (1)\) defined as the intersection of the following subsets:
  \[\begin{aligned}     &Q =  { \left \{ \alpha   \mid   \alpha   \sqsubseteq   \mathsf{eval} \right \} }   \\     &R =  { \left \{ \alpha   \mid   \forall   { \left [ c', e', n \right ]} ~  (e  \mapsto ^n c', e')  \land   { \left ( \alpha (e'; g,  \mathsf{ret} (w)) { \downarrow } \right )}   \to   \varphi (c' +  \alpha (e'; g,  \mathsf{ret} (w))) \right \} }    \end{aligned}\]   It suffices to show that \(\mathsf{eval}   \in  P\). We have that \(P\) is admissible, and we proceed by fixed-point induction. Suppose that \(\alpha   \in  P\). We need to show that \(\Phi _ \mathsf{eval} ( \alpha )  \in  P\), where \(\Phi _ \mathsf{eval}\) is the characteristic functional of \(\mathsf{eval}\) (\cref{subsec:comp-sem}). It's immediate that \(\Phi _ \mathsf{eval} ( \alpha )  \in  Q\). It remains to show that it is also contained in \(R\). So suppose that \(e  \mapsto ^n c', e'\) and \(\Phi _ \mathsf{eval} ( \alpha )(e'; g,  \mathsf{ret} (w)) { \downarrow }\). We want to show that \(\varphi (c' +  \Phi _ \mathsf{eval} ( \alpha )(e'; g,  \mathsf{ret} (w)))\). We proceed by cases on \(\mathsf{out} (e')\).  
   \begin{enumerate}
    \item{If \(\mathsf{out} (e') =  \mathsf{inl}   \cdot   \star\), then we know that \(e' =  \mathsf{ret} (v)\) for some \(v :  \mathsf{U} \mathsf{F} 1\). Stepping the operational semantics, we have that \(\mathsf{ret} (v); g  \mapsto  0, g~v\), and by definition of the computational semantics \(\Phi _ \mathsf{eval} ( \alpha )(e'; g,  \mathsf{ret} (w)) = 0  \boxplus   \alpha (g~v,  \mathsf{ret} (w)) =  \alpha (g~v,  \mathsf{ret} (w))\). Since we assumed \(\alpha   \in  Q  \iff   \alpha   \sqsubseteq   \mathsf{eval}\), we also have \(\mathsf{eval} (g~v,  \mathsf{ret} (w)) { \downarrow }\), and since \(\mathbb{C}\) is discrete we have \(\alpha (g~v,  \mathsf{ret} (w)) =  \mathsf{eval} (g~v,  \mathsf{ret} (w))\). Recalling the premise and the fact that \(\mathsf{eval} (e,  \mathsf{ret} (v)) = c'\), we may conclude that \(\varphi (c' +  \mathsf{eval} (g~v,  \mathsf{ret} (w)))\), which is what we needed to show.}    \item{Otherwise, \(\mathsf{out} (e') =  \mathsf{inr}   \cdot  (c'', e'')\) for some \(e'' :  \mathsf{U} \mathsf{F} 1\), and we have that \(e';g  \mapsto  c'', e'';g\). By definition of the computational semantics, this means that \(\Phi _ \mathsf{eval} ( \alpha )(e'; g,  \mathsf{ret} (w)) = c''  \boxplus   \alpha (e''; g,  \mathsf{ret} (w))\). Since we assumed that \(c''  \boxplus   \alpha (e''; g,  \mathsf{ret} (w)) { \downarrow }\), we can use the laws of the derived algebra (\cref{prop:cost-alg-laws}) to deduce that \(\alpha (e''; g,  \mathsf{ret} (w)) { \downarrow }\) as well, and so by the assumption that \(\alpha   \in  R  \subseteq  P\), we have that \(\varphi (c' + c'' +  \alpha (e'';g,  \mathsf{ret} (w)))\) holds, which is what we needed to show. }  
\end{enumerate} 
\end{proof}

\PropProfAssoc* 

\begin{proof}
  In one direction, we show that \(\mathsf{profile} ((e; g); i) { \downarrow }\) implies \(\mathsf{profile} (e; ( \lambda  v.~g~v; i)) { \downarrow }\) and both denote identical costs. Consider the \(\Sigma\)-predicate \(\varphi\) such that \(\varphi (c)\) if and only if \(\mathsf{profile} (e; ( \lambda  v.~g~v; h)) = c\). Suppose that \(\mathsf{eval} (e; g,  \mathsf{ret} (w)) { \downarrow }\) and \(\mathsf{profile} (i~w) { \downarrow }\). By computational induction on sequencing~\cref{prop:comp-ind-seq}, it suffices to show that \(\mathsf{profile} (e; ( \lambda  v.~g~v; h)) =  \mathsf{eval} (e; g,  \mathsf{ret} (w)) +  \mathsf{profile} (i~w)\). Applying computational induction on \(\mathsf{eval} (e; g,  \mathsf{ret} (w)) { \downarrow }\), we further suppose that \(\mathsf{eval} (e,  \mathsf{ret} (v)) { \downarrow }\) and \(\mathsf{eval} (g~v,  \mathsf{ret} (w)) { \downarrow }\) and aim to show that \(\mathsf{profile} (e; ( \lambda  v.~g~v; h)) =  \mathsf{eval} (e,  \mathsf{ret} (v)) +  \mathsf{eval} (g~v,  \mathsf{ret} (w)) +  \mathsf{profile} (i~w)\). 
  \begin{enumerate}
    \item{We claim that \(\mathsf{profile} (e; ( \lambda  v.~g~v; i)) { \downarrow }\). By the big-step semantics of profiling~\cref{prop:prof-seq}, it suffices to show that \(\mathsf{eval} (e,  \mathsf{ret} (v)) { \downarrow }\) for some \(v\) and \(\mathsf{profile} (g~v; i) { \downarrow }\). The former follows from our assumption; for the latter, it suffices to show that \(\mathsf{eval} (g~v,  \mathsf{ret} (w)) { \downarrow }\) and \(\mathsf{profile} (i~w) { \downarrow }\), both of which follow from assumptions. }    \item{Given that \(\mathsf{profile} (e; ( \lambda  v.~g~v; i)) { \downarrow }\), we may apply computational induction again: supposing that \(\mathsf{eval} (e,  \mathsf{ret} (v')) { \downarrow }\) and \(\mathsf{profile} (g~v; i) { \downarrow }\), we have to show that \(\mathsf{eval} (e,  \mathsf{ret} (v')) +  \mathsf{profile} (g~v; i) =  \mathsf{eval} (e,  \mathsf{ret} (v)) +  \mathsf{eval} (g~v,  \mathsf{ret} (w)) +  \mathsf{profile} (i~w)\), which follows from the uniqueness of evaluation~\cref{prop:eval-func} and big-step semantics of profiling~\cref{prop:prof-seq}. 
    }  
\end{enumerate}  \par{In the other direction, suppose that \(\mathsf{profile} (e; ( \lambda  v.~g~v; i)) { \downarrow }\). It suffices to show that \(\mathsf{profile} ((e; g); i) { \downarrow }\). By computational induction, we may assume that \(\mathsf{eval} (e,  \mathsf{ret} (v)) { \downarrow }\) and \(\mathsf{profile} (g~v; i) { \downarrow }\). Applying computational induction again, we can also assume that \(\mathsf{eval} (g~v,  \mathsf{ret} (w)) { \downarrow }\) and \(\mathsf{profile} (i~w) { \downarrow }\) for some \(w\). By the big-step semantics of profiling~\cref{prop:prof-seq}, it suffices to show that \(\mathsf{eval} (e; g,  \mathsf{ret} (w)) { \downarrow }\) and \(\mathsf{profile} (i~w) { \downarrow }\). The latter is our assumption, and the former follows from the big-step semantics of evaluation~\cref{prop:eval-seq}.}
\end{proof}

\begin{restatable}{proposition}{PropCompIndApp}\label[prop]{prop:comp-ind-app}
  The following is valid: 
  \begin{mathpar}
  \inferrule{
    \all{e} \kw{eval}(f, \lambda e)\supp{} \land \kw{eval}(e[v], \kw{ret}(w))\supp{} \to \varphi(\kw{eval}(f, \lambda e) + \kw{eval}(e[v], \kw{ret}(w)))
  }{
    \kw{eval}(f~v, \kw{ret}(w))\supp{} \to \varphi(\kw{eval}(f~v, \kw{ret}(w)))
  }
  \end{mathpar}
\end{restatable}

\begin{restatable}{proposition}{PropCommAppSeq}\label[prop]{prop:comm-app-seq}
  We have that \(\mathsf{eval} ((e; g)~w, z) =  \mathsf{eval} (e;  \lambda  v.~g~v~w, z)\). 
\end{restatable}

\section{Soundness of the denotational semantics}

\PropSoundStep* 

\begin{proof}
  By induction on the derivation of \(e  \mapsto  c,e'\). 
\end{proof}

\ThmSound* 

\begin{proof}
  Consider the following subset:
  \[\begin{aligned}     P =  { \left \{ \alpha   \mid   \forall   { \left [ e \right ]} ~   \alpha (e, v) { \downarrow }   \to   \llbracket   e   \rrbracket  =  \mathsf{eval} (e, v)  \boxplus   \llbracket   v   \rrbracket \right \} }  
   \end{aligned}\]  \par{Because \(\llbracket   e   \rrbracket  =  \mathsf{eval} (e, v)  \boxplus   \llbracket   v   \rrbracket\) is a \(\Sigma\)-proposition, we see that \(P\) is an admissible subset. Suppose that \(\alpha   \in  P\) and \(\Phi _{ \mathsf{eval} }( \alpha )(e, v) { \downarrow }\). We need to show that \(\llbracket   e   \rrbracket  =  \mathsf{eval} (e, v)  \boxplus   \llbracket   v   \rrbracket\). We proceed by cases on \(\mathsf{out} (e)\).}  \begin{enumerate}
    \item{If \(e  \mapsto  c',e'\), then by the soundness of the one step relation~\cref{prop:sound-step}, it suffices to show that \(c'  \boxplus   \llbracket   e'   \rrbracket  = c'  \boxplus   \mathsf{eval} (e', v)  \boxplus   \llbracket   v   \rrbracket\), which follows from the assumption, noting that \(\Phi _{ \mathsf{eval} }( \alpha )(e, v) { \downarrow }\) implies \(\alpha (e, v) { \downarrow }\).}    \item{Otherwise, we have that \(e = v\), and so the result holds since \(\mathsf{eval} (e, e) = 0\).}  
\end{enumerate}
\end{proof}

\section{Proofs for the computational adequacy property}\label{app:adequacy}

\begin{proposition}\label[prop]{prop:head-exp}
  If \(d  \lhd _X e\) and \(e'  \mapsto  c, e\), then \(c  \boxplus  d  \lhd _X e'\). 
\end{proposition}

\begin{proof}
  By induction on \(X\), using the laws of the cost algebra\cref{prop:cost-alg-laws}. 
\end{proof}

\subsection{Admissibility}

\begin{proposition}\label[prop]{prop:free-adm}
  We have that \(-  \lhd _{ \U{\F{A}} } e\) is an admissible subset of \(\kw{T}(A)\).
\end{proposition}

\begin{proof}
  By \cref{cor:admissible} it suffices to show downward closure and closure under $\bot$ and synthetic $\omega$-joins. 
  \begin{enumerate}
    \item By definition this means to show $f^\sharp(\bot) \le \kw{profile}(e; g)$ for all \(f  \mathrel{(A \Rightarrow \kw{adq})} g\). But this holds since $f^\sharp(\bot) = \bot$ and $\bot$ is the least element of $\kw{T}(A)$.
    \item Let $d$ be a synthetic $\omega$-chain such that $d_i \lhd_{\U{\F{A}}} e$ for all $i : \omega$. We want to show that $\sup d \lhd_{\U{\F{A}}} e$, which is to show that $f^\sharp(\sup d) \le \kw{profile}(e; g)$ for all \(f \mathrel{(A \Rightarrow \kw{adq})}  g\). Since $f^\sharp(\sup d) = \sup(f^\sharp \circ d)$, this means to show $\sup(f^\sharp \circ d) \le \kw{profile}(e; g)$. By the universal property of the synthetic $\omega$-join, it suffices to show $f^\sharp(d_i) \le \kw{profile}(e; g)$ for all $i:\omega$, but this is the assumption.
    \item Fixing $d' \le d \lhd_{\U{\F{A}}} e$, we need to show that $d' \lhd_{\U{\F{A}}} e$. Suppose that \(f \mathrel{(A \Rightarrow \kw{adq})}  g\). We need to show $f^\sharp(d') \le \kw{profile}(e; g)$. By the characterization of the order on lifts \cref{prop:lift-order}, we suppose $f^\sharp(d')\supp$ and show that $\kw{profile}(e; g)\supp$ and that $f^\sharp(d') = \kw{profile}(e; g)$. By assumption we know $d' = \eta_{\mathbb{L}}(a')$ and $f(a') = c$ for some $a : \sem{A}$ and $c : \P \vee \mathbb{C}$. Since $d' \le d$, we know $d' = \eta_{\mathbb{L}}(a)$ for some $a$ such that $a' \le a$. Consequently, we have $c = f(a') \le f(a)$, but since $\P \vee \mathbb{C}$ is discrete (by \cref{sec:den-sem} and the argument in \cref{prop:comp-ind-seq}), we have $f(a) = c = f(a')$. Thus by the assumption that $d \lhd_{\U{\F{A}}} e$, we have $f^{\sharp}(d) = c \le \kw{profile}(e; g)$. Again by the discreteness of $\P \vee \mathbb{C}$ we have that $f^\sharp(d') = f(a') = c = \kw{profile}(e; g)$, as required.
  \end{enumerate}
\end{proof}

\begin{proposition}\label[prop]{prop:sup-func}
  Suprema of synthetic $\omega$-chains in function spaces are computed pointwise.
\end{proposition}

\begin{proposition}\label[prop]{prop:func-adm}
  Given that \(-  \lhd _{ \mathsf{U} X } e\) is admissible for all \(e :  \mathsf{U} X\), we have that \(-  \lhd _{ \mathsf{U} (A  \to  X) } e\) is as well.
\end{proposition}

\begin{proof}
  Again we show downward closure and closure under $\bot$ and $\sup$. 
  \begin{enumerate}
    \item Because \(\bot (a) =  \bot\), we have that \(\bot   \lhd _{ \mathsf{U} (A  \to  X) } e\) by the assumption that $\bot \lhd_{\U{X}} e$ for all $e$. 
    \item Suppose that \(f_i  \lhd _{ \mathsf{U} (A  \to  X) } e\). We need to show that \(\bigvee  f  \lhd _{ \mathsf{U} (A  \to  X) } e\). Suppose that \(a  \lhd _A b\). We need to show that \(( \bigvee  f)~a  \lhd _{ \mathsf{U} X } e~b\). This follows the fact that synthetic \(\omega\)-joins in function spaces are computed pointwise and the assumption that \(-  \lhd _{ \mathsf{U} X } e~b\) is closed under $\sup$. 
    \item Fix $f' \le f \lhd_{\U{(A \to X)}} e$. To show that $f' \lhd_{\U{(A \to X)}} e$, suppose that $a \lhd_A b$. We need to show that $f'~a \lhd_{\U{X}} e~b$. By the premise, we have that \(-  \lhd _{ \mathsf{U} X } e~b\) is a lower set, so it suffices to show $f'~a \le f~a \lhd_{\U{X}} e~b$, which follow from the assumptions $f' \le f$ and $f \lhd_{\U{(A \to X)}} e$. 
  \end{enumerate} 
\end{proof}

\begin{proposition}\label[prop]{prop:formal-adm}
  Given \(e :  \U{X}\), we have that \(-  \lhd _{ \U{X} } e\) is an admissible subset of \(\sem{\U{X}}\). 
\end{proposition}

\begin{proof}
 By \cref{prop:free-adm,prop:func-adm}.  
\end{proof}

\subsection{Fundamental lemma}

We give the representative cases of the proof by induction on the derivation of terms. 

\begin{lemma}\label[lem]{lem:fllr-ret}
  If \(a  \lhd _A v\), then \(\eta _ \mathsf{T} (a)  \lhd _{ \mathsf{U} \mathsf{F} A }  \mathsf{ret} (v)\). 
\end{lemma}

\begin{proof}
  Let \(f  \mathrel{( \lhd _A  \Rightarrow {\kw{adq}} )}  g\). We need to show that \((f^ \sharp ( \eta _ \mathsf{T} (a)))  \mathrel{\kw{adq}}  ( \mathsf{ret} (v); g)\). Computing the denotational semantics and applying \cref{prop:head-exp}, it suffices to show that \((f~a) \mathrel{\kw{adq}} (g~v)\), which follows from our assumption. 
\end{proof}

\begin{lemma}\label[lem]{lem:fllr-seq}
  If \(d  \lhd _{ \mathsf{U} \mathsf{F} A } e\) and \(f  \lhd _{ \mathsf{U} (A  \to  X) } g\), then \(f^ \sharp (d)  \lhd _{ \mathsf{U} X } e; g\).
\end{lemma}

\begin{proof}
  \par{By induction on \(X\).}  \begin{enumerate}
    \item{If \(X =  \mathsf{F} B\), let \(h  \mathrel{( \lhd _B  \Rightarrow   \mathsf{adq} )}  i\). We need to show that \(h^ \sharp (f^ \sharp (d))  \mathrel{\mathsf{adq}}  (e; g); i\). Computing the denotational semantics and using the fact that we may reassociate sequences (\cref{prop:prof-assoc}), it suffices to show \(((h^ \sharp   \circ  f)^ \sharp (d))  \mathrel{\mathsf{adq}}  (e; ( \lambda  v.~g~v; i))\). By the assumption that \(d  \lhd _{ \mathsf{U} \mathsf{F} A } e\), it suffices to show that for all \(a  \lhd _A v\), we have that \((h^ \sharp (f~a))  \mathrel{\mathsf{adq}}  (g~v; i)\), which follows directly from the assumptions that \(f  \lhd _{ \mathsf{U} (A  \to  X) } g\) and \(h  \mathrel{( \lhd _B  \Rightarrow   \mathsf{adq} )}  i\). 
    }    \item{If \(X = B  \to  Y\), suppose that \(b  \lhd _B v\). We need to show that \((f^ \sharp (d))~b  \lhd _{ \mathsf{U} Y } (e; g)~v\). Unraveling the denotational semantics and the computational semantics (using \cref{prop:comm-app-seq}), it suffices to show \(( \lambda  d.~f~d~b)^ \sharp ~d  \lhd _{ \mathsf{U} Y } (e;  \lambda  d.~g~d~v)\), which follows from the inductive hypothesis and the assumption that \(f  \lhd _{ \mathsf{U} (A  \to  (B  \to  Y)) } g\). }
\end{enumerate}
\end{proof}

\begin{lemma}\label[lem]{lem:fllr-step}
  If \(d  \lhd _X e\), then \(c  \boxplus  d  \lhd _X  \mathsf{step} ^c(e)\). 
\end{lemma}

\begin{proof}
  Since \(\mathsf{step} ^c(e)  \mapsto  c, e\), the result holds by \cref{prop:head-exp}.
\end{proof}

\ThmFLLR* 

\begin{proof}
  By \cref{lem:fllr-ret,lem:fllr-seq,lem:fllr-step,prop:formal-adm}. 
\end{proof}

\ThmNoninterference* 

\begin{proof}
  Let $c : \mathbb{C}$ and $d : \mathbb{C}$ be the costs denoted by $\kw{eval}(e~x, \kw{ret}(v))$ and $\kw{eval}(e~y, \kw{ret}(u))$. By soundness \cref{thm:sound} and laws of the derived algebra \cref{prop:cost-alg-laws}, we have that $\sem{e~x} = c \costmap \eta_\mathbb{T}(\sem{v}) = \eta_\mathbb{L}(c, \sem{v})$ and similarly $\sem{e~y} =\costmap \eta_\mathbb{T}(\sem{u}) = \eta_\mathbb{L}(d, \sem{u})$. It suffices to show that $\sem{v} =_2 \sem{u}$. Because 2 is a purely extensional type (as required by \cref{def:axioms}), we may assume that $\P$ holds. By assumption and soundness, we have $\sem{x} = \sem{y}$, and so $\sem{e~x} = \eta_\mathbb{L}(c, \sem{v}) = \eta_\mathbb{L}(d, \sem{u}) = \sem{e~y}$, which means that $\sem{v} = \sem{u}$ since $c = d$ as elements of a purely intensional type $\mathbb{C}$. 
\end{proof}


\end{document}
